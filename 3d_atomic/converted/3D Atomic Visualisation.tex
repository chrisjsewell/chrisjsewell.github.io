

	\documentclass[10pt,parskip=half,
	toc=sectionentrywithdots,
	bibliography=totocnumbered,
	captions=tableheading,numbers=noendperiod]{scrartcl}
%\usepackage{polyglossia}
%\setmainlanguage{portuguese}
%\DeclareTextCommandDefault{\nobreakspace}{\leavevmode\nobreak\ } 
\usepackage[portuguese]{babel}

    \usepackage[T1]{fontenc} % Nicer default font (+ math font) than Computer Modern for most use cases
    \usepackage{mathpazo}
    \usepackage{graphicx}
    \usepackage[skip=3pt]{caption}    
    \usepackage{adjustbox} % Used to constrain images to a maximum size 
    \usepackage[table]{xcolor} % Allow colors to be defined
    \usepackage{enumerate} % Needed for markdown enumerations to work
    \usepackage{amsmath} % Equations
    \usepackage{amssymb} % Equations
    \usepackage{textcomp} % defines textquotesingle
    % Hack from http://tex.stackexchange.com/a/47451/13684:
    \AtBeginDocument{%
        \def\PYZsq{\textquotesingle}% Upright quotes in Pygmentized code
    }
    \usepackage{upquote} % Upright quotes for verbatim code
    \usepackage{eurosym} % defines \euro
    \usepackage[mathletters]{ucs} % Extended unicode (utf-8) support
    \usepackage[utf8x]{inputenc} % Allow utf-8 characters in the tex document
    \usepackage{fancyvrb} % verbatim replacement that allows latex
    \usepackage{grffile} % extends the file name processing of package graphics 
                         % to support a larger range 
    % The hyperref package gives us a pdf with properly built
    % internal navigation ('pdf bookmarks' for the table of contents,
    % internal cross-reference links, web links for URLs, etc.)
    \usepackage{hyperref}
    \usepackage{longtable} % longtable support required by pandoc >1.10
    \usepackage{booktabs}  % table support for pandoc > 1.12.2
    \usepackage[inline]{enumitem} % IRkernel/repr support (it uses the enumerate* environment)
    \usepackage[normalem]{ulem} % ulem is needed to support strikethroughs (\sout)
                                % normalem makes italics be italics, not underlines

    \usepackage{translations}
	\usepackage{microtype} % improves the spacing between words and letters
	\usepackage{placeins} % placement of figures
    % could use \usepackage[section]{placeins} but placing in subsection in command section
	% Places the float at precisely the location in the LaTeX code (with H)
	\usepackage{float}
	\usepackage[colorinlistoftodos,obeyFinal,textwidth=.8in]{todonotes} % to mark to-dos
	% number figures, tables and equations by section
	\usepackage{chngcntr}
	% header/footer
	\usepackage[footsepline=0.25pt]{scrlayer-scrpage}

	% bibliography formatting
	\usepackage[numbers, square, super, sort&compress]{natbib}
	% hyperlink doi's
	\usepackage{doi}	

    % define a code float
    \usepackage{newfloat} % to define a new float types
    \DeclareFloatingEnvironment[
        fileext=frm,placement={!ht},
        within=section,name=Code]{codecell}
    \DeclareFloatingEnvironment[
        fileext=frm,placement={!ht},
        within=section,name=Text]{textcell}
    \DeclareFloatingEnvironment[
        fileext=frm,placement={!ht},
        within=section,name=Text]{errorcell}

    \usepackage{listings} % a package for wrapping code in a box
    \usepackage[framemethod=tikz]{mdframed} % to fram code

% Pygments definitions

\makeatletter
\def\PY@reset{\let\PY@it=\relax \let\PY@bf=\relax%
    \let\PY@ul=\relax \let\PY@tc=\relax%
    \let\PY@bc=\relax \let\PY@ff=\relax}
\def\PY@tok#1{\csname PY@tok@#1\endcsname}
\def\PY@toks#1+{\ifx\relax#1\empty\else%
    \PY@tok{#1}\expandafter\PY@toks\fi}
\def\PY@do#1{\PY@bc{\PY@tc{\PY@ul{%
    \PY@it{\PY@bf{\PY@ff{#1}}}}}}}
\def\PY#1#2{\PY@reset\PY@toks#1+\relax+\PY@do{#2}}

\expandafter\def\csname PY@tok@w\endcsname{\def\PY@tc##1{\textcolor[rgb]{0.73,0.73,0.73}{##1}}}
\expandafter\def\csname PY@tok@c\endcsname{\let\PY@it=\textit\def\PY@tc##1{\textcolor[rgb]{0.25,0.50,0.50}{##1}}}
\expandafter\def\csname PY@tok@cp\endcsname{\def\PY@tc##1{\textcolor[rgb]{0.74,0.48,0.00}{##1}}}
\expandafter\def\csname PY@tok@k\endcsname{\let\PY@bf=\textbf\def\PY@tc##1{\textcolor[rgb]{0.00,0.50,0.00}{##1}}}
\expandafter\def\csname PY@tok@kp\endcsname{\def\PY@tc##1{\textcolor[rgb]{0.00,0.50,0.00}{##1}}}
\expandafter\def\csname PY@tok@kt\endcsname{\def\PY@tc##1{\textcolor[rgb]{0.69,0.00,0.25}{##1}}}
\expandafter\def\csname PY@tok@o\endcsname{\def\PY@tc##1{\textcolor[rgb]{0.40,0.40,0.40}{##1}}}
\expandafter\def\csname PY@tok@ow\endcsname{\let\PY@bf=\textbf\def\PY@tc##1{\textcolor[rgb]{0.67,0.13,1.00}{##1}}}
\expandafter\def\csname PY@tok@nb\endcsname{\def\PY@tc##1{\textcolor[rgb]{0.00,0.50,0.00}{##1}}}
\expandafter\def\csname PY@tok@nf\endcsname{\def\PY@tc##1{\textcolor[rgb]{0.00,0.00,1.00}{##1}}}
\expandafter\def\csname PY@tok@nc\endcsname{\let\PY@bf=\textbf\def\PY@tc##1{\textcolor[rgb]{0.00,0.00,1.00}{##1}}}
\expandafter\def\csname PY@tok@nn\endcsname{\let\PY@bf=\textbf\def\PY@tc##1{\textcolor[rgb]{0.00,0.00,1.00}{##1}}}
\expandafter\def\csname PY@tok@ne\endcsname{\let\PY@bf=\textbf\def\PY@tc##1{\textcolor[rgb]{0.82,0.25,0.23}{##1}}}
\expandafter\def\csname PY@tok@nv\endcsname{\def\PY@tc##1{\textcolor[rgb]{0.10,0.09,0.49}{##1}}}
\expandafter\def\csname PY@tok@no\endcsname{\def\PY@tc##1{\textcolor[rgb]{0.53,0.00,0.00}{##1}}}
\expandafter\def\csname PY@tok@nl\endcsname{\def\PY@tc##1{\textcolor[rgb]{0.63,0.63,0.00}{##1}}}
\expandafter\def\csname PY@tok@ni\endcsname{\let\PY@bf=\textbf\def\PY@tc##1{\textcolor[rgb]{0.60,0.60,0.60}{##1}}}
\expandafter\def\csname PY@tok@na\endcsname{\def\PY@tc##1{\textcolor[rgb]{0.49,0.56,0.16}{##1}}}
\expandafter\def\csname PY@tok@nt\endcsname{\let\PY@bf=\textbf\def\PY@tc##1{\textcolor[rgb]{0.00,0.50,0.00}{##1}}}
\expandafter\def\csname PY@tok@nd\endcsname{\def\PY@tc##1{\textcolor[rgb]{0.67,0.13,1.00}{##1}}}
\expandafter\def\csname PY@tok@s\endcsname{\def\PY@tc##1{\textcolor[rgb]{0.73,0.13,0.13}{##1}}}
\expandafter\def\csname PY@tok@sd\endcsname{\let\PY@it=\textit\def\PY@tc##1{\textcolor[rgb]{0.73,0.13,0.13}{##1}}}
\expandafter\def\csname PY@tok@si\endcsname{\let\PY@bf=\textbf\def\PY@tc##1{\textcolor[rgb]{0.73,0.40,0.53}{##1}}}
\expandafter\def\csname PY@tok@se\endcsname{\let\PY@bf=\textbf\def\PY@tc##1{\textcolor[rgb]{0.73,0.40,0.13}{##1}}}
\expandafter\def\csname PY@tok@sr\endcsname{\def\PY@tc##1{\textcolor[rgb]{0.73,0.40,0.53}{##1}}}
\expandafter\def\csname PY@tok@ss\endcsname{\def\PY@tc##1{\textcolor[rgb]{0.10,0.09,0.49}{##1}}}
\expandafter\def\csname PY@tok@sx\endcsname{\def\PY@tc##1{\textcolor[rgb]{0.00,0.50,0.00}{##1}}}
\expandafter\def\csname PY@tok@m\endcsname{\def\PY@tc##1{\textcolor[rgb]{0.40,0.40,0.40}{##1}}}
\expandafter\def\csname PY@tok@gh\endcsname{\let\PY@bf=\textbf\def\PY@tc##1{\textcolor[rgb]{0.00,0.00,0.50}{##1}}}
\expandafter\def\csname PY@tok@gu\endcsname{\let\PY@bf=\textbf\def\PY@tc##1{\textcolor[rgb]{0.50,0.00,0.50}{##1}}}
\expandafter\def\csname PY@tok@gd\endcsname{\def\PY@tc##1{\textcolor[rgb]{0.63,0.00,0.00}{##1}}}
\expandafter\def\csname PY@tok@gi\endcsname{\def\PY@tc##1{\textcolor[rgb]{0.00,0.63,0.00}{##1}}}
\expandafter\def\csname PY@tok@gr\endcsname{\def\PY@tc##1{\textcolor[rgb]{1.00,0.00,0.00}{##1}}}
\expandafter\def\csname PY@tok@ge\endcsname{\let\PY@it=\textit}
\expandafter\def\csname PY@tok@gs\endcsname{\let\PY@bf=\textbf}
\expandafter\def\csname PY@tok@gp\endcsname{\let\PY@bf=\textbf\def\PY@tc##1{\textcolor[rgb]{0.00,0.00,0.50}{##1}}}
\expandafter\def\csname PY@tok@go\endcsname{\def\PY@tc##1{\textcolor[rgb]{0.53,0.53,0.53}{##1}}}
\expandafter\def\csname PY@tok@gt\endcsname{\def\PY@tc##1{\textcolor[rgb]{0.00,0.27,0.87}{##1}}}
\expandafter\def\csname PY@tok@err\endcsname{\def\PY@bc##1{\setlength{\fboxsep}{0pt}\fcolorbox[rgb]{1.00,0.00,0.00}{1,1,1}{\strut ##1}}}
\expandafter\def\csname PY@tok@kc\endcsname{\let\PY@bf=\textbf\def\PY@tc##1{\textcolor[rgb]{0.00,0.50,0.00}{##1}}}
\expandafter\def\csname PY@tok@kd\endcsname{\let\PY@bf=\textbf\def\PY@tc##1{\textcolor[rgb]{0.00,0.50,0.00}{##1}}}
\expandafter\def\csname PY@tok@kn\endcsname{\let\PY@bf=\textbf\def\PY@tc##1{\textcolor[rgb]{0.00,0.50,0.00}{##1}}}
\expandafter\def\csname PY@tok@kr\endcsname{\let\PY@bf=\textbf\def\PY@tc##1{\textcolor[rgb]{0.00,0.50,0.00}{##1}}}
\expandafter\def\csname PY@tok@bp\endcsname{\def\PY@tc##1{\textcolor[rgb]{0.00,0.50,0.00}{##1}}}
\expandafter\def\csname PY@tok@fm\endcsname{\def\PY@tc##1{\textcolor[rgb]{0.00,0.00,1.00}{##1}}}
\expandafter\def\csname PY@tok@vc\endcsname{\def\PY@tc##1{\textcolor[rgb]{0.10,0.09,0.49}{##1}}}
\expandafter\def\csname PY@tok@vg\endcsname{\def\PY@tc##1{\textcolor[rgb]{0.10,0.09,0.49}{##1}}}
\expandafter\def\csname PY@tok@vi\endcsname{\def\PY@tc##1{\textcolor[rgb]{0.10,0.09,0.49}{##1}}}
\expandafter\def\csname PY@tok@vm\endcsname{\def\PY@tc##1{\textcolor[rgb]{0.10,0.09,0.49}{##1}}}
\expandafter\def\csname PY@tok@sa\endcsname{\def\PY@tc##1{\textcolor[rgb]{0.73,0.13,0.13}{##1}}}
\expandafter\def\csname PY@tok@sb\endcsname{\def\PY@tc##1{\textcolor[rgb]{0.73,0.13,0.13}{##1}}}
\expandafter\def\csname PY@tok@sc\endcsname{\def\PY@tc##1{\textcolor[rgb]{0.73,0.13,0.13}{##1}}}
\expandafter\def\csname PY@tok@dl\endcsname{\def\PY@tc##1{\textcolor[rgb]{0.73,0.13,0.13}{##1}}}
\expandafter\def\csname PY@tok@s2\endcsname{\def\PY@tc##1{\textcolor[rgb]{0.73,0.13,0.13}{##1}}}
\expandafter\def\csname PY@tok@sh\endcsname{\def\PY@tc##1{\textcolor[rgb]{0.73,0.13,0.13}{##1}}}
\expandafter\def\csname PY@tok@s1\endcsname{\def\PY@tc##1{\textcolor[rgb]{0.73,0.13,0.13}{##1}}}
\expandafter\def\csname PY@tok@mb\endcsname{\def\PY@tc##1{\textcolor[rgb]{0.40,0.40,0.40}{##1}}}
\expandafter\def\csname PY@tok@mf\endcsname{\def\PY@tc##1{\textcolor[rgb]{0.40,0.40,0.40}{##1}}}
\expandafter\def\csname PY@tok@mh\endcsname{\def\PY@tc##1{\textcolor[rgb]{0.40,0.40,0.40}{##1}}}
\expandafter\def\csname PY@tok@mi\endcsname{\def\PY@tc##1{\textcolor[rgb]{0.40,0.40,0.40}{##1}}}
\expandafter\def\csname PY@tok@il\endcsname{\def\PY@tc##1{\textcolor[rgb]{0.40,0.40,0.40}{##1}}}
\expandafter\def\csname PY@tok@mo\endcsname{\def\PY@tc##1{\textcolor[rgb]{0.40,0.40,0.40}{##1}}}
\expandafter\def\csname PY@tok@ch\endcsname{\let\PY@it=\textit\def\PY@tc##1{\textcolor[rgb]{0.25,0.50,0.50}{##1}}}
\expandafter\def\csname PY@tok@cm\endcsname{\let\PY@it=\textit\def\PY@tc##1{\textcolor[rgb]{0.25,0.50,0.50}{##1}}}
\expandafter\def\csname PY@tok@cpf\endcsname{\let\PY@it=\textit\def\PY@tc##1{\textcolor[rgb]{0.25,0.50,0.50}{##1}}}
\expandafter\def\csname PY@tok@c1\endcsname{\let\PY@it=\textit\def\PY@tc##1{\textcolor[rgb]{0.25,0.50,0.50}{##1}}}
\expandafter\def\csname PY@tok@cs\endcsname{\let\PY@it=\textit\def\PY@tc##1{\textcolor[rgb]{0.25,0.50,0.50}{##1}}}

\def\PYZbs{\char`\\}
\def\PYZus{\char`\_}
\def\PYZob{\char`\{}
\def\PYZcb{\char`\}}
\def\PYZca{\char`\^}
\def\PYZam{\char`\&}
\def\PYZlt{\char`\<}
\def\PYZgt{\char`\>}
\def\PYZsh{\char`\#}
\def\PYZpc{\char`\%}
\def\PYZdl{\char`\$}
\def\PYZhy{\char`\-}
\def\PYZsq{\char`\'}
\def\PYZdq{\char`\"}
\def\PYZti{\char`\~}
% for compatibility with earlier versions
\def\PYZat{@}
\def\PYZlb{[}
\def\PYZrb{]}
\makeatother

% ANSI colors
\definecolor{ansi-black}{HTML}{3E424D}
\definecolor{ansi-black-intense}{HTML}{282C36}
\definecolor{ansi-red}{HTML}{E75C58}
\definecolor{ansi-red-intense}{HTML}{B22B31}
\definecolor{ansi-green}{HTML}{00A250}
\definecolor{ansi-green-intense}{HTML}{007427}
\definecolor{ansi-yellow}{HTML}{DDB62B}
\definecolor{ansi-yellow-intense}{HTML}{B27D12}
\definecolor{ansi-blue}{HTML}{208FFB}
\definecolor{ansi-blue-intense}{HTML}{0065CA}
\definecolor{ansi-magenta}{HTML}{D160C4}
\definecolor{ansi-magenta-intense}{HTML}{A03196}
\definecolor{ansi-cyan}{HTML}{60C6C8}
\definecolor{ansi-cyan-intense}{HTML}{258F8F}
\definecolor{ansi-white}{HTML}{C5C1B4}
\definecolor{ansi-white-intense}{HTML}{A1A6B2}

% commands and environments needed by pandoc snippets
% extracted from the output of `pandoc -s`
\providecommand{\tightlist}{%
  \setlength{\itemsep}{0pt}\setlength{\parskip}{0pt}}
\DefineVerbatimEnvironment{Highlighting}{Verbatim}{commandchars=\\\{\}}
% Add ',fontsize=\small' for more characters per line
\newenvironment{Shaded}{}{}
\newcommand{\KeywordTok}[1]{\textcolor[rgb]{0.00,0.44,0.13}{\textbf{{#1}}}}
\newcommand{\DataTypeTok}[1]{\textcolor[rgb]{0.56,0.13,0.00}{{#1}}}
\newcommand{\DecValTok}[1]{\textcolor[rgb]{0.25,0.63,0.44}{{#1}}}
\newcommand{\BaseNTok}[1]{\textcolor[rgb]{0.25,0.63,0.44}{{#1}}}
\newcommand{\FloatTok}[1]{\textcolor[rgb]{0.25,0.63,0.44}{{#1}}}
\newcommand{\CharTok}[1]{\textcolor[rgb]{0.25,0.44,0.63}{{#1}}}
\newcommand{\StringTok}[1]{\textcolor[rgb]{0.25,0.44,0.63}{{#1}}}
\newcommand{\CommentTok}[1]{\textcolor[rgb]{0.38,0.63,0.69}{\textit{{#1}}}}
\newcommand{\OtherTok}[1]{\textcolor[rgb]{0.00,0.44,0.13}{{#1}}}
\newcommand{\AlertTok}[1]{\textcolor[rgb]{1.00,0.00,0.00}{\textbf{{#1}}}}
\newcommand{\FunctionTok}[1]{\textcolor[rgb]{0.02,0.16,0.49}{{#1}}}
\newcommand{\RegionMarkerTok}[1]{{#1}}
\newcommand{\ErrorTok}[1]{\textcolor[rgb]{1.00,0.00,0.00}{\textbf{{#1}}}}
\newcommand{\NormalTok}[1]{{#1}}

% Additional commands for more recent versions of Pandoc
\newcommand{\ConstantTok}[1]{\textcolor[rgb]{0.53,0.00,0.00}{{#1}}}
\newcommand{\SpecialCharTok}[1]{\textcolor[rgb]{0.25,0.44,0.63}{{#1}}}
\newcommand{\VerbatimStringTok}[1]{\textcolor[rgb]{0.25,0.44,0.63}{{#1}}}
\newcommand{\SpecialStringTok}[1]{\textcolor[rgb]{0.73,0.40,0.53}{{#1}}}
\newcommand{\ImportTok}[1]{{#1}}
\newcommand{\DocumentationTok}[1]{\textcolor[rgb]{0.73,0.13,0.13}{\textit{{#1}}}}
\newcommand{\AnnotationTok}[1]{\textcolor[rgb]{0.38,0.63,0.69}{\textbf{\textit{{#1}}}}}
\newcommand{\CommentVarTok}[1]{\textcolor[rgb]{0.38,0.63,0.69}{\textbf{\textit{{#1}}}}}
\newcommand{\VariableTok}[1]{\textcolor[rgb]{0.10,0.09,0.49}{{#1}}}
\newcommand{\ControlFlowTok}[1]{\textcolor[rgb]{0.00,0.44,0.13}{\textbf{{#1}}}}
\newcommand{\OperatorTok}[1]{\textcolor[rgb]{0.40,0.40,0.40}{{#1}}}
\newcommand{\BuiltInTok}[1]{{#1}}
\newcommand{\ExtensionTok}[1]{{#1}}
\newcommand{\PreprocessorTok}[1]{\textcolor[rgb]{0.74,0.48,0.00}{{#1}}}
\newcommand{\AttributeTok}[1]{\textcolor[rgb]{0.49,0.56,0.16}{{#1}}}
\newcommand{\InformationTok}[1]{\textcolor[rgb]{0.38,0.63,0.69}{\textbf{\textit{{#1}}}}}
\newcommand{\WarningTok}[1]{\textcolor[rgb]{0.38,0.63,0.69}{\textbf{\textit{{#1}}}}}

% Define a nice break command that doesn't care if a line doesn't already
% exist.
\def\br{\hspace*{\fill} \\* }

% Math Jax compatability definitions
\def\gt{>}
\def\lt{<}

    % Colors for the hyperref package
    \definecolor{urlcolor}{rgb}{0,.145,.698}
    \definecolor{linkcolor}{rgb}{.71,0.21,0.01}
    \definecolor{citecolor}{rgb}{.12,.54,.11}

\DeclareTranslationFallback{Author}{Author}
\DeclareTranslation{Portuges}{Author}{Autor}

\DeclareTranslationFallback{List of Codes}{List of Codes}
\DeclareTranslation{Catalan}{List of Codes}{Llista de Codis}
\DeclareTranslation{Danish}{List of Codes}{Liste over Koder}
\DeclareTranslation{German}{List of Codes}{Liste der Codes}
\DeclareTranslation{Spanish}{List of Codes}{Lista de C\'{o}digos}
\DeclareTranslation{French}{List of Codes}{Liste des Codes}
\DeclareTranslation{Italian}{List of Codes}{Elenco dei Codici}
\DeclareTranslation{Dutch}{List of Codes}{Lijst van Codes}
\DeclareTranslation{Portuges}{List of Codes}{Lista de C\'{o}digos} 

\DeclareTranslationFallback{Supervisors}{Supervisors}
\DeclareTranslation{Catalan}{Supervisors}{Supervisors}
\DeclareTranslation{Danish}{Supervisors}{Vejledere}
\DeclareTranslation{German}{Supervisors}{Vorgesetzten}
\DeclareTranslation{Spanish}{Supervisors}{Supervisores}
\DeclareTranslation{French}{Supervisors}{Superviseurs}
\DeclareTranslation{Italian}{Supervisors}{Le autorit\`{a} di vigilanza}
\DeclareTranslation{Dutch}{Supervisors}{supervisors}
\DeclareTranslation{Portuguese}{Supervisors}{Supervisores} 

\definecolor{codegreen}{rgb}{0,0.6,0}
\definecolor{codegray}{rgb}{0.5,0.5,0.5}
\definecolor{codepurple}{rgb}{0.58,0,0.82}
\definecolor{backcolour}{rgb}{0.95,0.95,0.95}

\lstdefinestyle{mystyle}{
    commentstyle=\color{codegreen},
    keywordstyle=\color{magenta},
    numberstyle=\tiny\color{codegray},
    stringstyle=\color{codepurple},
    basicstyle=\ttfamily,
    breakatwhitespace=false,         
    keepspaces=true,                 
    numbers=left,                    
    numbersep=10pt,                  
    showspaces=false,                
    showstringspaces=false,
    showtabs=false,                  
    tabsize=2,
    breaklines=true,
    literate={\-}{}{0\discretionary{-}{}{-}},
  postbreak=\mbox{\textcolor{red}{$\hookrightarrow$}\space},
}

\lstset{style=mystyle} 

\surroundwithmdframed[
  hidealllines=true,
  backgroundcolor=backcolour,
  innerleftmargin=0pt,
  innerrightmargin=0pt,
  innertopmargin=0pt,
  innerbottommargin=0pt]{lstlisting}

 % Used to adjust the document margins
\usepackage{geometry}
\geometry{tmargin=1in,bmargin=1in,lmargin=1in,rmargin=1in,
nohead,includefoot,footskip=25pt}
% you can use showframe option to check the margins visually 

	% ensure new section starts on new page
	\addtokomafont{section}{\clearpage}

    % Prevent overflowing lines due to hard-to-break entities
    \sloppy 

    % Setup hyperref package
    \hypersetup{
      breaklinks=true,  % so long urls are correctly broken across lines
      colorlinks=true,
      urlcolor=urlcolor,
      linkcolor=linkcolor,
      citecolor=citecolor,
      }

    % ensure figures are placed within subsections
    \makeatletter
    \AtBeginDocument{%
      \expandafter\renewcommand\expandafter\subsection\expandafter
        {\expandafter\@fb@secFB\subsection}%
      \newcommand\@fb@secFB{\FloatBarrier
        \gdef\@fb@afterHHook{\@fb@topbarrier \gdef\@fb@afterHHook{}}}%
      \g@addto@macro\@afterheading{\@fb@afterHHook}%
      \gdef\@fb@afterHHook{}%
    }
    \makeatother

	% number figures, tables and equations by section
	\usepackage{chngcntr}
	\counterwithout{figure}{section}
	\counterwithout{table}{section}
	\counterwithout{equation}{section}
	\makeatletter
	\@addtoreset{table}{section}
	\@addtoreset{figure}{section}
	\@addtoreset{equation}{section}
	\makeatother
	\renewcommand\thetable{\thesection.\arabic{table}}
	\renewcommand\thefigure{\thesection.\arabic{figure}}
	\renewcommand\theequation{\thesection.\arabic{equation}}

        % set global options for float placement
        \makeatletter
          \providecommand*\setfloatlocations[2]{\@namedef{fps@#1}{#2}}
        \makeatother

    % align captions to left (indented)
	\captionsetup{justification=raggedright,
	singlelinecheck=false,format=hang,labelfont={it,bf}} 

	% shift footer down so space between separation line
	\ModifyLayer[addvoffset=.6ex]{scrheadings.foot.odd}
	\ModifyLayer[addvoffset=.6ex]{scrheadings.foot.even}
	\ModifyLayer[addvoffset=.6ex]{scrheadings.foot.oneside}
	\ModifyLayer[addvoffset=.6ex]{plain.scrheadings.foot.odd}
	\ModifyLayer[addvoffset=.6ex]{plain.scrheadings.foot.even}
	\ModifyLayer[addvoffset=.6ex]{plain.scrheadings.foot.oneside}
	\pagestyle{scrheadings}
	\clearscrheadfoot{}
	\ifoot{\leftmark}
	\renewcommand{\sectionmark}[1]{\markleft{\thesection\ #1}}
	\ofoot{\pagemark}
	\cfoot{}

% clereref must be loaded after anything that changes the referencing system
\usepackage{cleveref}
\creflabelformat{equation}{#2#1#3}

% make the code float work with cleverref
\crefname{codecell}{code}{codes}
\Crefname{codecell}{code}{codes}
% make the text float work with cleverref
\crefname{textcell}{text}{texts}
\Crefname{textcell}{text}{texts}
% make the text float work with cleverref
\crefname{errorcell}{error}{errors}
\Crefname{errorcell}{error}{errors}

	\begin{document}

		\begin{titlepage}

	\begin{center}

	\vspace*{1cm}

	\Huge\textbf{Visualisation of 3D atomic and electronic data}

	\vspace{0.5cm}

	\vspace{1.5cm}

	\begin{minipage}{0.8\textwidth}   
		\begin{center}  
		\begin{minipage}{0.39\textwidth}
		\begin{flushleft} \Large
		\emph{\GetTranslation{Author}:}\\Chris Sewell\\
		\end{flushleft}
		\end{minipage}
		\hspace{\fill}
		\begin{minipage}{0.39\textwidth}
		\begin{flushright} \Large\emph{\GetTranslation{Supervisors}:} \\
			  one
		\end{flushright}
		\end{minipage}
		\end{center}   
	\end{minipage}

	\vfill

	\begin{minipage}{0.8\textwidth}
	\begin{center}
	\end{center} 
	\end{minipage}

	\vspace{0.8cm}
		  \LARGE{Imperial College London}\\
		  \LARGE{South Kensington, London}\\

	\vspace{0.4cm}

	\today

	\end{center}
	\end{titlepage}

		\begingroup
    \let\cleardoublepage\relax
    \let\clearpage\relax\tableofcontents\listoffigures\listoftables\listof{codecell}{\GetTranslation{List of Codes}}
    \endgroup

\section{Introduction}\label{introduction}

With the improvements in Jupyter Notebook, allowing for the synergy of
browser-side javascript and client-side python coding, it is becoming
possible to replicate the functionality of standalone atomic
visualisation packages (such as
\href{https://ovito.org/index.php/about}{ovito}). The added benefits
this approach brings is:

\begin{itemize}
\tightlist
\item
  Greater control and flexibility in the analysis and visualisation
  process
\item
  Fully autonomous replication of the analysis and visualisation
\item
  Better documentation of the analysis and visualisation
\end{itemize}

The goal of this notebook is to show a method for:

\begin{enumerate}
\def\labelenumi{\arabic{enumi}.}
\tightlist
\item
  reading/creating atomic configurations
\item
  visualising these in a Jupyter Notebook
\item
  Adding dynamic controls
\item
  Overlaying electronic level data (probability/spin densities)
\item
  Distributing the output
\end{enumerate}

\section{Creating Atomic
Configurations}\label{creating-atomic-configurations}

The \href{http://pymatgen.org/}{pymatgen} package offers a means to
create/manipulate atomic configurations with repeating boundary
conditions.

For this demonstration, we would like to select structures from a folder
of cif (crystallographic information files). This is achieved by writing
a \emph{plugin} for
\href{https://github.com/chrisjsewell/jsonextended}{jsonextended}, which
is a package parsing file types into a json format and subsequent
manipulation.

\begin{lstlisting}[language={},postbreak={},numbers=none,xrightmargin=7pt,belowskip=5pt,aboveskip=5pt,breakindent=0pt]
FeS_greigite.cif: {...}
FeS_mackinawite.cif: {...}
FeS_marcasite.cif: {...}
FeS_pyrite.cif: {...}
FeS_pyrrhotite_4C_c2c.cif: {...}
FeS_troilite.cif: {...}
Fe_bcc.cif: {...}
S_8alpha_fddd.cif: {...}
S_8beta_p21c.cif: {...}

\end{lstlisting}

For each cif, we can now access a pymatgen structure:

\begin{lstlisting}[language={},postbreak={},numbers=none,xrightmargin=7pt,breakindent=0pt,aboveskip=5pt,belowskip=5pt]
Structure Summary
Lattice
    abc : 3.6735000000000002 3.6735000000000002 5.0327999999999999
 angles : 90.0 90.0 90.0
 volume : 67.91563420380001
      A : 3.6735000000000002 0.0 2.2493700083339009e-16
      B : -2.2493700083339009e-16 3.6735000000000002 2.2493700083339009e-16
      C : 0.0 0.0 5.0327999999999999
PeriodicSite: Fe (0.0000, 0.0000, 0.0000) [0.0000, 0.0000, 0.0000]
PeriodicSite: Fe (1.8367, 1.8368, 0.0000) [0.5000, 0.5000, 0.0000]
PeriodicSite: S (-0.0000, 1.8368, 1.3095) [0.0000, 0.5000, 0.2602]
PeriodicSite: S (1.8368, 0.0000, 3.7233) [0.5000, 0.0000, 0.7398]
\end{lstlisting}

\section{Preparation for
visualisation}\label{preparation-for-visualisation}

A visualisation requires the configuration to contain some additional
information, including the atom shape (e.g. sphere radius) and texture
(e.g. sphere color).

Therefore, it will be helpful to create a view agnostic (i.e.
independendant of any specific graphics package) representation of all
elements we wish to visualise. We do this by deconstructing the pymatgen
structure and applying a mapping of atomic number to radius/color, using
a pre-constructed csv table.

    \begin{table}[H]\caption{The first rows of the atomic data lookup.}

        \centering
		\begin{adjustbox}{max width=\textwidth}
        \begin{tabular}{lrrrrrrlrrrrl}
\toprule
{} &  Blue &  ElAffinity &  ElNeg &  Green &  Ionization &   Mass &       Name &   RBO &  RCov &  RVdW &   Red & Symbol \\
\midrule
  &       &             &        &        &             &        &            &       &       &       &       &        \\
1 &  0.75 &        0.75 &   2.20 &   0.75 &       13.60 &   1.01 &   Hydrogen &  0.31 &  0.31 &  1.10 &  0.75 &      H \\
2 &  1.00 &        0.00 &   0.00 &   1.00 &       24.59 &   4.00 &     Helium &  0.28 &  0.28 &  1.40 &  0.85 &     He \\
3 &  1.00 &        0.62 &   0.98 &   0.50 &        5.39 &   6.94 &    Lithium &  1.28 &  1.28 &  1.81 &  0.80 &     Li \\
4 &  0.00 &        0.00 &   1.57 &   1.00 &        9.32 &   9.01 &  Beryllium &  0.96 &  0.96 &  1.53 &  0.76 &     Be \\
5 &  0.71 &        0.28 &   2.04 &   0.71 &        8.30 &  10.81 &      Boron &  0.84 &  0.84 &  1.92 &  1.00 &      B \\
\bottomrule
\end{tabular}

		\end{adjustbox}
        \end{table}

\begin{lstlisting}[language={},postbreak={},numbers=none,xrightmargin=7pt,belowskip=5pt,aboveskip=5pt,breakindent=0pt,escapechar=\%]
%\textcolor{ansi-blue}{mackinawite\_Fe}%: 
  %\textcolor{ansi-blue}{cell\_vectors}%: 
    %\textcolor{ansi-blue}{a}%: [ 3.67350000e+00 0.00000000e+00 2.24937001e-16]
    %\textcolor{ansi-blue}{b}%: [ -2.24937001e-16 3.67350000e+00 2.24937001e-16]
    %\textcolor{ansi-blue}{c}%: [ 0. 0. 5.0328]
  %\textcolor{ansi-blue}{centre}%: [ 1.83675 1.83675 2.5164 ]
  %\textcolor{ansi-blue}{color}%: rgb(224,102,51)
  %\textcolor{ansi-blue}{coords}%: [[ 0. 0. 0.], [ 1.83675000e+00 1.83675000e+00 
                   2.24937001e-16]]
  %\textcolor{ansi-blue}{label}%: Fe
  %\textcolor{ansi-blue}{radius}%: 1.32
  %\textcolor{ansi-blue}{transparency}%: 1.0
  %\textcolor{ansi-blue}{type}%: scatter
  %\textcolor{ansi-blue}{visible}%: [True, True]
%\textcolor{ansi-blue}{mackinawite\_S}%: 
  %\textcolor{ansi-blue}{cell\_vectors}%: 
    %\textcolor{ansi-blue}{a}%: [ 3.67350000e+00 0.00000000e+00 2.24937001e-16]
    %\textcolor{ansi-blue}{b}%: [ -2.24937001e-16 3.67350000e+00 2.24937001e-16]
    %\textcolor{ansi-blue}{c}%: [ 0. 0. 5.0328]
  %\textcolor{ansi-blue}{centre}%: [ 1.83675 1.83675 2.5164 ]
  %\textcolor{ansi-blue}{color}%: rgb(178,178,0)
  %\textcolor{ansi-blue}{coords}%: [[ -1.12468500e-16 1.83675000e+00 
                   1.30953456e+00], [ 1.83675 0. 3.72326544]]
  %\textcolor{ansi-blue}{label}%: S
  %\textcolor{ansi-blue}{radius}%: 1.05
  %\textcolor{ansi-blue}{transparency}%: 1.0
  %\textcolor{ansi-blue}{type}%: scatter
  %\textcolor{ansi-blue}{visible}%: [True, True]

\end{lstlisting}

Since the representation is in a JSON format, it makes it very easy to
extend to new types of elements. Note that we group atoms with the same
visual representations, rather than specifying each atom separately.
This is because it will be more efficient for the rendering process (see
\href{http://www.ianww.com/blog/2012/11/04/optimizing-three-dot-js-performance-simulating-tens-of-thousands-of-independent-moving-objects/}{here}
for an explanation). Here we do this by atomic number, but equally it
could be done by symmetry equivalence or another metric.

\subsection{Geometry Manipulation}\label{geometry-manipulation}

We will also likely want to:

\begin{itemize}
\tightlist
\item
  create a supercell of the configuration
\item
  orientate the configuration in a convenient manner in the cartesian
  coordinate space
\item
  slice into the configuration
\end{itemize}

We can group these operations into a class, which is extensible to more
geometric operation and element types.

\begin{lstlisting}[language={},postbreak={},numbers=none,xrightmargin=7pt,belowskip=5pt,aboveskip=5pt,breakindent=0pt,escapechar=\%]
%\textcolor{ansi-blue}{mackinawite\_Fe}%: 
  %\textcolor{ansi-blue}{cell\_vectors}%: 
    %\textcolor{ansi-blue}{a}%: [ -6.74811003e-16 -1.10205000e+01 -1.47911420e-31]
    %\textcolor{ansi-blue}{b}%: [ 3.67350000e+00 -1.37734189e-32 -4.93038066e-32]
    %\textcolor{ansi-blue}{c}%: [ 3.08170121e-16 -3.08170121e-16 5.03280000e+00]
  %\textcolor{ansi-blue}{centre}%: [ 0. 0. 0.]
  %\textcolor{ansi-blue}{color}%: rgb(224,102,51)
  %\textcolor{ansi-blue}{coords}%: [[-1.83675 5.51025 -2.5164 ], [ 0. 3.6735 
                   -2.5164], [-1.83675 1.83675 -2.5164 ], [ 
                   -2.22044605e-16 0.00000000e+00 -2.51640000e+00], 
                   [-1.83675 -1.83675 -2.5164 ], [ -4.44089210e-16 
                   -3.67350000e+00 -2.51640000e+00]]
  %\textcolor{ansi-blue}{label}%: Fe
  %\textcolor{ansi-blue}{radius}%: 1.32
  %\textcolor{ansi-blue}{transparency}%: 1.0
  %\textcolor{ansi-blue}{type}%: scatter
  %\textcolor{ansi-blue}{visible}%: [True, True, True, True, True, True]
%\textcolor{ansi-blue}{mackinawite\_S}%: 
  %\textcolor{ansi-blue}{cell\_vectors}%: 
    %\textcolor{ansi-blue}{a}%: [ -6.74811003e-16 -1.10205000e+01 -1.47911420e-31]
    %\textcolor{ansi-blue}{b}%: [ 3.67350000e+00 -1.37734189e-32 -4.93038066e-32]
    %\textcolor{ansi-blue}{c}%: [ 3.08170121e-16 -3.08170121e-16 5.03280000e+00]
  %\textcolor{ansi-blue}{centre}%: [ 0. 0. 0.]
  %\textcolor{ansi-blue}{color}%: rgb(178,178,0)
  %\textcolor{ansi-blue}{coords}%: [[ 2.22044605e-16 5.51025000e+00 
                   -1.20686544e+00], [-1.83675 3.6735 1.20686544], 
                   [ 0. 1.83675 -1.20686544], [ -1.83675000e+00 
                   -8.88178420e-16 1.20686544e+00], [ 
                   -2.22044605e-16 -1.83675000e+00 
                   -1.20686544e+00], [-1.83675 -3.6735 1.20686544]]
  %\textcolor{ansi-blue}{label}%: S
  %\textcolor{ansi-blue}{radius}%: 1.05
  %\textcolor{ansi-blue}{transparency}%: 1.0
  %\textcolor{ansi-blue}{type}%: scatter
  %\textcolor{ansi-blue}{visible}%: [True, False, True, False, True, False]

\end{lstlisting}

\section{Visualising in the Jupyter
Notebook}\label{visualising-in-the-jupyter-notebook}

To create 3D renderings of the configuration, we will use
\href{http://ipyvolume.readthedocs.io}{ipyvolume} and its implementation
of the model/view pattern.

\begin{lstlisting}[language={},postbreak={},numbers=none,xrightmargin=7pt,breakindent=0pt,aboveskip=5pt,belowskip=5pt]
A Jupyter Widget
\end{lstlisting}

The rendering can also be captured as a screenshot or saved as an
image/html. We shall discuss in Section \ref{analysis-distribution} how
this can be utilised for to distribute the analysis.

\begin{lstlisting}[language={},postbreak={},numbers=none,xrightmargin=7pt,breakindent=0pt,aboveskip=5pt,belowskip=5pt]
A Jupyter Widget
\end{lstlisting}

\begin{figure}[H]\begin{center}\adjustimage{max size={0.9\linewidth}{0.9\paperheight},height=0.3\paperheight}{3D Atomic Visualisation_files/output_29_0.png}\end{center}\caption{an example of an ipyvolume scatter plot}\label{fig:ipyvol1}
    \end{figure}

\section{Adding Dynamic Controls}\label{adding-dynamic-controls}

\href{http://ipyvolume.readthedocs.io}{ipyvolume} utilises the
\href{http://ipywidgets.readthedocs.io}{ipywidgets} framework and thus
it is relatively trivial to set up dynamic controls.

\begin{lstlisting}[language={},postbreak={},numbers=none,xrightmargin=7pt,breakindent=0pt,aboveskip=5pt,belowskip=5pt]
A Jupyter Widget
\end{lstlisting}

We can bundle these in with the original container to create a bespoke
GUI.

\begin{lstlisting}[language={},postbreak={},numbers=none,xrightmargin=7pt,breakindent=0pt,aboveskip=5pt,belowskip=5pt]
A Jupyter Widget
\end{lstlisting}

\begin{lstlisting}[language={},postbreak={},numbers=none,xrightmargin=7pt,breakindent=0pt,aboveskip=5pt,belowskip=5pt]
A Jupyter Widget
\end{lstlisting}

\begin{figure}[H]\begin{center}\adjustimage{max size={0.9\linewidth}{0.9\paperheight},height=0.3\paperheight}{3D Atomic Visualisation_files/output_37_0.png}\end{center}\caption{an example of an ipyvolume scatter plot (with bespoke controls)}\label{fig:ipyvol2}
    \end{figure}

\section{Overlaying electronic level
data}\label{overlaying-electronic-level-data}

\emph{Ab initio} quantum simulation packages can compute electronic/spin
densities (to accompany the nuclei positions) in the form of a
discretized 3D cube. These can be overlayed onto the nuclei, by
\href{https://en.wikipedia.org/wiki/Volume_rendering}{volume rendering}
or \href{https://en.wikipedia.org/wiki/Isosurface}{isosurface} methods.

\subsection{Data Parsing}\label{data-parsing}

Taking the \href{http://www.crystal.unito.it/index.php}{CRYSTAL} program
as an example, output from the electronic density is principally output
into two files; one that contains the lattice vectors and nuclei
coordinates and one that contains a data cube of the electronic density,
with axis relating to the cell vectors. We can write parser plugins for
both these files:

\begin{lstlisting}[language={},postbreak={},numbers=none,xrightmargin=7pt,belowskip=5pt,aboveskip=5pt,breakindent=0pt]
ech3.out: 
  structure: Full Formula (Si2) Reduced Formula: Si abc   :   3.832519 
                      3.832519   3.832519 angles:  60.000000  60.000000 
                      60.000000 Sites (2)   #  SP           a         b 
                      c ---  ----  --------  --------  --------   0  Si 
                      0.125092  0.125092  0.125092   1  Si    0.874908  0.875277 
                      0.874908
ech3_dat.prop3d: 
  charge_density: np.array((100, 100, 100), min=2.68E-03, max=5.36E+02)
  da_vec: [ 0. 0.051729 0.051729]
  db_vec: [ 0.051729 0. 0.051729]
  dc_vec: [ 0.051729 0.051729 0. ]
  na:    100
  nb:    100
  nc:    100
  o_vec: [0.0, 0.0, 0.0]

\end{lstlisting}

\subsection{Visualisation Preparation}\label{visualisation-preparation}

We then, follow the same process as for atoms; converting to a common
structure and adding geometric manipulation functions for this data
type.

\begin{lstlisting}[language={},postbreak={},numbers=none,xrightmargin=7pt,belowskip=5pt,aboveskip=5pt,breakindent=0pt,escapechar=\%]
%\textcolor{ansi-blue}{Silicon Charge}%: 
  %\textcolor{ansi-blue}{cell\_vectors}%: 
    %\textcolor{ansi-blue}{a}%: [ 0. 2.71 2.71]
    %\textcolor{ansi-blue}{b}%: [ 2.71 0. 2.71]
    %\textcolor{ansi-blue}{c}%: [ 2.71 2.71 0. ]
  %\textcolor{ansi-blue}{centre}%: [ 2.71 2.71 2.71]
  %\textcolor{ansi-blue}{dcube}%: np.array((100, 100, 100), min=2.68E-03, max=5.36E+02)
  %\textcolor{ansi-blue}{slices}%: []
  %\textcolor{ansi-blue}{type}%: volume
%\textcolor{ansi-blue}{Silicon\_Si}%: 
  %\textcolor{ansi-blue}{cell\_vectors}%: 
    %\textcolor{ansi-blue}{a}%: [ 0. 2.71 2.71]
    %\textcolor{ansi-blue}{b}%: [ 2.71 0. 2.71]
    %\textcolor{ansi-blue}{c}%: [ 2.71 2.71 0. ]
  %\textcolor{ansi-blue}{centre}%: [ 2.71 2.71 2.71]
  %\textcolor{ansi-blue}{color}%: rgb(127,153,153)
  %\textcolor{ansi-blue}{coords}%: [[ 0.678 0.678 0.678], [ 4.743 4.742 4.743]]
  %\textcolor{ansi-blue}{label}%: Si
  %\textcolor{ansi-blue}{radius}%: 1.11
  %\textcolor{ansi-blue}{transparency}%: 1.0
  %\textcolor{ansi-blue}{type}%: scatter
  %\textcolor{ansi-blue}{visible}%: [True, True]

\end{lstlisting}

\begin{lstlisting}[language={},postbreak={},numbers=none,xrightmargin=7pt,belowskip=5pt,aboveskip=5pt,breakindent=0pt]
cell_vectors: 
  a: [ 0. 10.84 10.84]
  b: [ 10.84 0. 10.84]
  c: [ 10.84 10.84 0. ]
centre: [ 0. 0. 0.]
dcube:  np.array((400, 400, 400), min=2.68E-03, max=5.36E+02)
slices: [([ 0. 0. 1.], None, -4, None)]
type:   volume

\end{lstlisting}

\subsection{Visualisation}\label{visualisation}

For ipyvolume, at present, the volume data must a cube of equal
dimensions. Therefore, we use the cell vectors to transform the data
cube into cartesian coordinates, such that voxels (cube sections)
outside of the cell volume are set as np.nan values. We also resize the
discretisation of the cube to an appropriate size for the renderer to
handle.

Additionally, for ipyvolume (in its current state), there can only be
one volume rendering per scene and it is assumed that the volumes bottom
left corner is at (0,0,0).

\begin{lstlisting}[language={},postbreak={},numbers=none,xrightmargin=7pt,belowskip=5pt,aboveskip=5pt,breakindent=0pt]
cell_vectors: 
  a: [ 0. 10.84 10.84]
  b: [ 10.84 0. 10.84]
  c: [ 10.84 10.84 0. ]
centre: [ 0. 0. 0.]
dcube:  np.array((400, 400, 400), min=2.68E-03, max=5.36E+02)
slices: [([ 0. 1. 1.], -2, 0, None)]
type:   volume

\end{lstlisting}

\begin{lstlisting}[language={},postbreak={},numbers=none,xrightmargin=7pt,breakindent=0pt,aboveskip=5pt,belowskip=5pt]
A Jupyter Widget
\end{lstlisting}

\begin{lstlisting}[language={},postbreak={},numbers=none,xrightmargin=7pt,breakindent=0pt,aboveskip=5pt,belowskip=5pt]
A Jupyter Widget
\end{lstlisting}

\begin{figure}[H]\begin{center}\adjustimage{max size={0.9\linewidth}{0.9\paperheight},height=0.3\paperheight}{3D Atomic Visualisation_files/output_58_0.png}\end{center}\caption{an example of an ipyvolume volume and scatter plot (with bespoke controls)}\label{fig:ipyvol3}
    \end{figure}

\section{2D Representations}\label{d-representations}

Because the data is stored in a representation agnostic manner, this
allows for the possibility of displaying the data in multiple ways. In
particular, for publication quality images we may want to create a 2D
representation of the scene.

\begin{lstlisting}[language={},postbreak={},numbers=none,xrightmargin=7pt,belowskip=5pt,aboveskip=5pt,breakindent=0pt]
mackinawite_Fe: 
  cell_vectors: {...}
  centre:       
  color:        
  coords:       
  label:        
  radius:       
  transparency: 
  type:         
  visible:      
mackinawite_S: 
  cell_vectors: {...}
  centre:       
  color:        
  coords:       
  label:        
  radius:       
  transparency: 
  type:         
  visible:      

\end{lstlisting}

\begin{lstlisting}[language={},postbreak={},numbers=none,xrightmargin=7pt,breakindent=0pt,aboveskip=5pt,belowskip=5pt]
dict_keys(['mackinawite_Fe', 'mackinawite_S'])
\end{lstlisting}

\begin{figure}[H]\begin{center}\adjustimage{max size={0.9\linewidth}{0.9\paperheight}}{3D Atomic Visualisation_files/output_62_1.pdf}\end{center}
    \end{figure}

\section{Analysis Distribution}\label{analysis-distribution}

\section{TODO}\label{todo}

\begin{itemize}
\tightlist
\item
  Finalise electronic density section

  \begin{itemize}
  \tightlist
  \item
    transfer parsing code into jsonextended plugin
  \item
    tidy rest of code and put into document
  \end{itemize}
\item
  Orthographic camera. Not yet implemented in ipyvolume, see
  \href{https://github.com/maartenbreddels/ipyvolume/issues/31}{this
  issue} for current status.
\item
  better control of spheres

  \begin{itemize}
  \tightlist
  \item
    radius rather than size
  \item
    more segments
  \item
    transparency level
  \end{itemize}
\item
  show lattice bounding box: parallelepiped wire frames
\item
  show nearest-neighbour coordination: polygons with vertices at
  nearest-neighbour positions (as shown in \cref{fig:nnpolygons})
\end{itemize}

\begin{figure}[H]\begin{center}\adjustimage{max size={0.9\linewidth}{0.9\paperheight},height=0.2\paperheight}{3D Atomic Visualisation_files/output_68_0.jpeg}\end{center}\caption{an example of nearest-neighbour polygons}\label{fig:nnpolygons}
    \end{figure}

\begin{itemize}
\tightlist
\item
  functional (browser side) controls, e.g. slider to translate/rotate
  point set. Not yet implemented in ipywidgets, see
  \href{https://github.com/jupyter-widgets/ipywidgets/issues/1109}{this
  issue} for current status.
\item
  volumes:

  \begin{itemize}
  \tightlist
  \item
    RuntimeWarning: invalid value encountered in true\_divide
    (serialize.py:43) presumably for (0,0,0) gradients
  \item
    rarely get artifact rendering
  \item
    isosurface rendering
  \item
    multiple volumes
  \item
    volumes with arbitrary centres
  \item
    rotating volumes
  \end{itemize}
\item
  fullscreen

  \begin{itemize}
  \tightlist
  \item
    fails to open if multiple views instantiated
  \item
    if volume is present, then the rendering becomes very low resolution
    and the volume disappears completely on exit
  \end{itemize}
\end{itemize}

	\end{document}

