

	\documentclass[10pt,parskip=half,
	toc=sectionentrywithdots,
	bibliography=totocnumbered,
	captions=tableheading,numbers=noendperiod]{scrartcl}
%\usepackage{polyglossia}
%\setmainlanguage{british}
%\DeclareTextCommandDefault{\nobreakspace}{\leavevmode\nobreak\ } 
\usepackage[british]{babel}

    \usepackage[T1]{fontenc} % Nicer default font (+ math font) than Computer Modern for most use cases
    \usepackage{mathpazo}
    \usepackage{graphicx}
    \usepackage[skip=3pt]{caption}    
    \usepackage{adjustbox} % Used to constrain images to a maximum size 
    \usepackage[table]{xcolor} % Allow colors to be defined
    \usepackage{enumerate} % Needed for markdown enumerations to work
    \usepackage{amsmath} % Equations
    \usepackage{amssymb} % Equations
    \usepackage{textcomp} % defines textquotesingle
    % Hack from http://tex.stackexchange.com/a/47451/13684:
    \AtBeginDocument{%
        \def\PYZsq{\textquotesingle}% Upright quotes in Pygmentized code
    }
    \usepackage{upquote} % Upright quotes for verbatim code
    \usepackage{eurosym} % defines \euro
    \usepackage[mathletters]{ucs} % Extended unicode (utf-8) support
    \usepackage[utf8x]{inputenc} % Allow utf-8 characters in the tex document
    \usepackage{fancyvrb} % verbatim replacement that allows latex
    \usepackage{grffile} % extends the file name processing of package graphics 
                         % to support a larger range 
    % The hyperref package gives us a pdf with properly built
    % internal navigation ('pdf bookmarks' for the table of contents,
    % internal cross-reference links, web links for URLs, etc.)
    \usepackage{hyperref}
    \usepackage{longtable} % longtable support required by pandoc >1.10
    \usepackage{booktabs}  % table support for pandoc > 1.12.2
    \usepackage[inline]{enumitem} % IRkernel/repr support (it uses the enumerate* environment)
    \usepackage[normalem]{ulem} % ulem is needed to support strikethroughs (\sout)
                                % normalem makes italics be italics, not underlines

    \usepackage{translations}
	\usepackage{microtype} % improves the spacing between words and letters
	\usepackage{placeins} % placement of figures
    % could use \usepackage[section]{placeins} but placing in subsection in command section
	% Places the float at precisely the location in the LaTeX code (with H)
	\usepackage{float}
	\usepackage[colorinlistoftodos,obeyFinal,textwidth=.8in]{todonotes} % to mark to-dos
	% number figures, tables and equations by section
	\usepackage{chngcntr}
	% header/footer
	\usepackage[footsepline=0.25pt]{scrlayer-scrpage}

	% bibliography formatting
	\usepackage[numbers, square, super, sort&compress]{natbib}
	% hyperlink doi's
	\usepackage{doi}

    % define a code float
    \usepackage{newfloat} % to define a new float types
    \DeclareFloatingEnvironment[
        fileext=frm,placement={!ht},
        within=section,name=Code]{codecell}
    \DeclareFloatingEnvironment[
        fileext=frm,placement={!ht},
        within=section,name=Text]{textcell}
    \DeclareFloatingEnvironment[
        fileext=frm,placement={!ht},
        within=section,name=Text]{errorcell}

    \usepackage{listings} % a package for wrapping code in a box
    \usepackage[framemethod=tikz]{mdframed} % to fram code

% Pygments definitions

\makeatletter
\def\PY@reset{\let\PY@it=\relax \let\PY@bf=\relax%
    \let\PY@ul=\relax \let\PY@tc=\relax%
    \let\PY@bc=\relax \let\PY@ff=\relax}
\def\PY@tok#1{\csname PY@tok@#1\endcsname}
\def\PY@toks#1+{\ifx\relax#1\empty\else%
    \PY@tok{#1}\expandafter\PY@toks\fi}
\def\PY@do#1{\PY@bc{\PY@tc{\PY@ul{%
    \PY@it{\PY@bf{\PY@ff{#1}}}}}}}
\def\PY#1#2{\PY@reset\PY@toks#1+\relax+\PY@do{#2}}

\expandafter\def\csname PY@tok@w\endcsname{\def\PY@tc##1{\textcolor[rgb]{0.73,0.73,0.73}{##1}}}
\expandafter\def\csname PY@tok@c\endcsname{\let\PY@it=\textit\def\PY@tc##1{\textcolor[rgb]{0.25,0.50,0.50}{##1}}}
\expandafter\def\csname PY@tok@cp\endcsname{\def\PY@tc##1{\textcolor[rgb]{0.74,0.48,0.00}{##1}}}
\expandafter\def\csname PY@tok@k\endcsname{\let\PY@bf=\textbf\def\PY@tc##1{\textcolor[rgb]{0.00,0.50,0.00}{##1}}}
\expandafter\def\csname PY@tok@kp\endcsname{\def\PY@tc##1{\textcolor[rgb]{0.00,0.50,0.00}{##1}}}
\expandafter\def\csname PY@tok@kt\endcsname{\def\PY@tc##1{\textcolor[rgb]{0.69,0.00,0.25}{##1}}}
\expandafter\def\csname PY@tok@o\endcsname{\def\PY@tc##1{\textcolor[rgb]{0.40,0.40,0.40}{##1}}}
\expandafter\def\csname PY@tok@ow\endcsname{\let\PY@bf=\textbf\def\PY@tc##1{\textcolor[rgb]{0.67,0.13,1.00}{##1}}}
\expandafter\def\csname PY@tok@nb\endcsname{\def\PY@tc##1{\textcolor[rgb]{0.00,0.50,0.00}{##1}}}
\expandafter\def\csname PY@tok@nf\endcsname{\def\PY@tc##1{\textcolor[rgb]{0.00,0.00,1.00}{##1}}}
\expandafter\def\csname PY@tok@nc\endcsname{\let\PY@bf=\textbf\def\PY@tc##1{\textcolor[rgb]{0.00,0.00,1.00}{##1}}}
\expandafter\def\csname PY@tok@nn\endcsname{\let\PY@bf=\textbf\def\PY@tc##1{\textcolor[rgb]{0.00,0.00,1.00}{##1}}}
\expandafter\def\csname PY@tok@ne\endcsname{\let\PY@bf=\textbf\def\PY@tc##1{\textcolor[rgb]{0.82,0.25,0.23}{##1}}}
\expandafter\def\csname PY@tok@nv\endcsname{\def\PY@tc##1{\textcolor[rgb]{0.10,0.09,0.49}{##1}}}
\expandafter\def\csname PY@tok@no\endcsname{\def\PY@tc##1{\textcolor[rgb]{0.53,0.00,0.00}{##1}}}
\expandafter\def\csname PY@tok@nl\endcsname{\def\PY@tc##1{\textcolor[rgb]{0.63,0.63,0.00}{##1}}}
\expandafter\def\csname PY@tok@ni\endcsname{\let\PY@bf=\textbf\def\PY@tc##1{\textcolor[rgb]{0.60,0.60,0.60}{##1}}}
\expandafter\def\csname PY@tok@na\endcsname{\def\PY@tc##1{\textcolor[rgb]{0.49,0.56,0.16}{##1}}}
\expandafter\def\csname PY@tok@nt\endcsname{\let\PY@bf=\textbf\def\PY@tc##1{\textcolor[rgb]{0.00,0.50,0.00}{##1}}}
\expandafter\def\csname PY@tok@nd\endcsname{\def\PY@tc##1{\textcolor[rgb]{0.67,0.13,1.00}{##1}}}
\expandafter\def\csname PY@tok@s\endcsname{\def\PY@tc##1{\textcolor[rgb]{0.73,0.13,0.13}{##1}}}
\expandafter\def\csname PY@tok@sd\endcsname{\let\PY@it=\textit\def\PY@tc##1{\textcolor[rgb]{0.73,0.13,0.13}{##1}}}
\expandafter\def\csname PY@tok@si\endcsname{\let\PY@bf=\textbf\def\PY@tc##1{\textcolor[rgb]{0.73,0.40,0.53}{##1}}}
\expandafter\def\csname PY@tok@se\endcsname{\let\PY@bf=\textbf\def\PY@tc##1{\textcolor[rgb]{0.73,0.40,0.13}{##1}}}
\expandafter\def\csname PY@tok@sr\endcsname{\def\PY@tc##1{\textcolor[rgb]{0.73,0.40,0.53}{##1}}}
\expandafter\def\csname PY@tok@ss\endcsname{\def\PY@tc##1{\textcolor[rgb]{0.10,0.09,0.49}{##1}}}
\expandafter\def\csname PY@tok@sx\endcsname{\def\PY@tc##1{\textcolor[rgb]{0.00,0.50,0.00}{##1}}}
\expandafter\def\csname PY@tok@m\endcsname{\def\PY@tc##1{\textcolor[rgb]{0.40,0.40,0.40}{##1}}}
\expandafter\def\csname PY@tok@gh\endcsname{\let\PY@bf=\textbf\def\PY@tc##1{\textcolor[rgb]{0.00,0.00,0.50}{##1}}}
\expandafter\def\csname PY@tok@gu\endcsname{\let\PY@bf=\textbf\def\PY@tc##1{\textcolor[rgb]{0.50,0.00,0.50}{##1}}}
\expandafter\def\csname PY@tok@gd\endcsname{\def\PY@tc##1{\textcolor[rgb]{0.63,0.00,0.00}{##1}}}
\expandafter\def\csname PY@tok@gi\endcsname{\def\PY@tc##1{\textcolor[rgb]{0.00,0.63,0.00}{##1}}}
\expandafter\def\csname PY@tok@gr\endcsname{\def\PY@tc##1{\textcolor[rgb]{1.00,0.00,0.00}{##1}}}
\expandafter\def\csname PY@tok@ge\endcsname{\let\PY@it=\textit}
\expandafter\def\csname PY@tok@gs\endcsname{\let\PY@bf=\textbf}
\expandafter\def\csname PY@tok@gp\endcsname{\let\PY@bf=\textbf\def\PY@tc##1{\textcolor[rgb]{0.00,0.00,0.50}{##1}}}
\expandafter\def\csname PY@tok@go\endcsname{\def\PY@tc##1{\textcolor[rgb]{0.53,0.53,0.53}{##1}}}
\expandafter\def\csname PY@tok@gt\endcsname{\def\PY@tc##1{\textcolor[rgb]{0.00,0.27,0.87}{##1}}}
\expandafter\def\csname PY@tok@err\endcsname{\def\PY@bc##1{\setlength{\fboxsep}{0pt}\fcolorbox[rgb]{1.00,0.00,0.00}{1,1,1}{\strut ##1}}}
\expandafter\def\csname PY@tok@kc\endcsname{\let\PY@bf=\textbf\def\PY@tc##1{\textcolor[rgb]{0.00,0.50,0.00}{##1}}}
\expandafter\def\csname PY@tok@kd\endcsname{\let\PY@bf=\textbf\def\PY@tc##1{\textcolor[rgb]{0.00,0.50,0.00}{##1}}}
\expandafter\def\csname PY@tok@kn\endcsname{\let\PY@bf=\textbf\def\PY@tc##1{\textcolor[rgb]{0.00,0.50,0.00}{##1}}}
\expandafter\def\csname PY@tok@kr\endcsname{\let\PY@bf=\textbf\def\PY@tc##1{\textcolor[rgb]{0.00,0.50,0.00}{##1}}}
\expandafter\def\csname PY@tok@bp\endcsname{\def\PY@tc##1{\textcolor[rgb]{0.00,0.50,0.00}{##1}}}
\expandafter\def\csname PY@tok@fm\endcsname{\def\PY@tc##1{\textcolor[rgb]{0.00,0.00,1.00}{##1}}}
\expandafter\def\csname PY@tok@vc\endcsname{\def\PY@tc##1{\textcolor[rgb]{0.10,0.09,0.49}{##1}}}
\expandafter\def\csname PY@tok@vg\endcsname{\def\PY@tc##1{\textcolor[rgb]{0.10,0.09,0.49}{##1}}}
\expandafter\def\csname PY@tok@vi\endcsname{\def\PY@tc##1{\textcolor[rgb]{0.10,0.09,0.49}{##1}}}
\expandafter\def\csname PY@tok@vm\endcsname{\def\PY@tc##1{\textcolor[rgb]{0.10,0.09,0.49}{##1}}}
\expandafter\def\csname PY@tok@sa\endcsname{\def\PY@tc##1{\textcolor[rgb]{0.73,0.13,0.13}{##1}}}
\expandafter\def\csname PY@tok@sb\endcsname{\def\PY@tc##1{\textcolor[rgb]{0.73,0.13,0.13}{##1}}}
\expandafter\def\csname PY@tok@sc\endcsname{\def\PY@tc##1{\textcolor[rgb]{0.73,0.13,0.13}{##1}}}
\expandafter\def\csname PY@tok@dl\endcsname{\def\PY@tc##1{\textcolor[rgb]{0.73,0.13,0.13}{##1}}}
\expandafter\def\csname PY@tok@s2\endcsname{\def\PY@tc##1{\textcolor[rgb]{0.73,0.13,0.13}{##1}}}
\expandafter\def\csname PY@tok@sh\endcsname{\def\PY@tc##1{\textcolor[rgb]{0.73,0.13,0.13}{##1}}}
\expandafter\def\csname PY@tok@s1\endcsname{\def\PY@tc##1{\textcolor[rgb]{0.73,0.13,0.13}{##1}}}
\expandafter\def\csname PY@tok@mb\endcsname{\def\PY@tc##1{\textcolor[rgb]{0.40,0.40,0.40}{##1}}}
\expandafter\def\csname PY@tok@mf\endcsname{\def\PY@tc##1{\textcolor[rgb]{0.40,0.40,0.40}{##1}}}
\expandafter\def\csname PY@tok@mh\endcsname{\def\PY@tc##1{\textcolor[rgb]{0.40,0.40,0.40}{##1}}}
\expandafter\def\csname PY@tok@mi\endcsname{\def\PY@tc##1{\textcolor[rgb]{0.40,0.40,0.40}{##1}}}
\expandafter\def\csname PY@tok@il\endcsname{\def\PY@tc##1{\textcolor[rgb]{0.40,0.40,0.40}{##1}}}
\expandafter\def\csname PY@tok@mo\endcsname{\def\PY@tc##1{\textcolor[rgb]{0.40,0.40,0.40}{##1}}}
\expandafter\def\csname PY@tok@ch\endcsname{\let\PY@it=\textit\def\PY@tc##1{\textcolor[rgb]{0.25,0.50,0.50}{##1}}}
\expandafter\def\csname PY@tok@cm\endcsname{\let\PY@it=\textit\def\PY@tc##1{\textcolor[rgb]{0.25,0.50,0.50}{##1}}}
\expandafter\def\csname PY@tok@cpf\endcsname{\let\PY@it=\textit\def\PY@tc##1{\textcolor[rgb]{0.25,0.50,0.50}{##1}}}
\expandafter\def\csname PY@tok@c1\endcsname{\let\PY@it=\textit\def\PY@tc##1{\textcolor[rgb]{0.25,0.50,0.50}{##1}}}
\expandafter\def\csname PY@tok@cs\endcsname{\let\PY@it=\textit\def\PY@tc##1{\textcolor[rgb]{0.25,0.50,0.50}{##1}}}

\def\PYZbs{\char`\\}
\def\PYZus{\char`\_}
\def\PYZob{\char`\{}
\def\PYZcb{\char`\}}
\def\PYZca{\char`\^}
\def\PYZam{\char`\&}
\def\PYZlt{\char`\<}
\def\PYZgt{\char`\>}
\def\PYZsh{\char`\#}
\def\PYZpc{\char`\%}
\def\PYZdl{\char`\$}
\def\PYZhy{\char`\-}
\def\PYZsq{\char`\'}
\def\PYZdq{\char`\"}
\def\PYZti{\char`\~}
% for compatibility with earlier versions
\def\PYZat{@}
\def\PYZlb{[}
\def\PYZrb{]}
\makeatother

% ANSI colors
\definecolor{ansi-black}{HTML}{3E424D}
\definecolor{ansi-black-intense}{HTML}{282C36}
\definecolor{ansi-red}{HTML}{E75C58}
\definecolor{ansi-red-intense}{HTML}{B22B31}
\definecolor{ansi-green}{HTML}{00A250}
\definecolor{ansi-green-intense}{HTML}{007427}
\definecolor{ansi-yellow}{HTML}{DDB62B}
\definecolor{ansi-yellow-intense}{HTML}{B27D12}
\definecolor{ansi-blue}{HTML}{208FFB}
\definecolor{ansi-blue-intense}{HTML}{0065CA}
\definecolor{ansi-magenta}{HTML}{D160C4}
\definecolor{ansi-magenta-intense}{HTML}{A03196}
\definecolor{ansi-cyan}{HTML}{60C6C8}
\definecolor{ansi-cyan-intense}{HTML}{258F8F}
\definecolor{ansi-white}{HTML}{C5C1B4}
\definecolor{ansi-white-intense}{HTML}{A1A6B2}

% commands and environments needed by pandoc snippets
% extracted from the output of `pandoc -s`
\providecommand{\tightlist}{%
  \setlength{\itemsep}{0pt}\setlength{\parskip}{0pt}}
\DefineVerbatimEnvironment{Highlighting}{Verbatim}{commandchars=\\\{\}}
% Add ',fontsize=\small' for more characters per line
\newenvironment{Shaded}{}{}
\newcommand{\KeywordTok}[1]{\textcolor[rgb]{0.00,0.44,0.13}{\textbf{{#1}}}}
\newcommand{\DataTypeTok}[1]{\textcolor[rgb]{0.56,0.13,0.00}{{#1}}}
\newcommand{\DecValTok}[1]{\textcolor[rgb]{0.25,0.63,0.44}{{#1}}}
\newcommand{\BaseNTok}[1]{\textcolor[rgb]{0.25,0.63,0.44}{{#1}}}
\newcommand{\FloatTok}[1]{\textcolor[rgb]{0.25,0.63,0.44}{{#1}}}
\newcommand{\CharTok}[1]{\textcolor[rgb]{0.25,0.44,0.63}{{#1}}}
\newcommand{\StringTok}[1]{\textcolor[rgb]{0.25,0.44,0.63}{{#1}}}
\newcommand{\CommentTok}[1]{\textcolor[rgb]{0.38,0.63,0.69}{\textit{{#1}}}}
\newcommand{\OtherTok}[1]{\textcolor[rgb]{0.00,0.44,0.13}{{#1}}}
\newcommand{\AlertTok}[1]{\textcolor[rgb]{1.00,0.00,0.00}{\textbf{{#1}}}}
\newcommand{\FunctionTok}[1]{\textcolor[rgb]{0.02,0.16,0.49}{{#1}}}
\newcommand{\RegionMarkerTok}[1]{{#1}}
\newcommand{\ErrorTok}[1]{\textcolor[rgb]{1.00,0.00,0.00}{\textbf{{#1}}}}
\newcommand{\NormalTok}[1]{{#1}}

% Additional commands for more recent versions of Pandoc
\newcommand{\ConstantTok}[1]{\textcolor[rgb]{0.53,0.00,0.00}{{#1}}}
\newcommand{\SpecialCharTok}[1]{\textcolor[rgb]{0.25,0.44,0.63}{{#1}}}
\newcommand{\VerbatimStringTok}[1]{\textcolor[rgb]{0.25,0.44,0.63}{{#1}}}
\newcommand{\SpecialStringTok}[1]{\textcolor[rgb]{0.73,0.40,0.53}{{#1}}}
\newcommand{\ImportTok}[1]{{#1}}
\newcommand{\DocumentationTok}[1]{\textcolor[rgb]{0.73,0.13,0.13}{\textit{{#1}}}}
\newcommand{\AnnotationTok}[1]{\textcolor[rgb]{0.38,0.63,0.69}{\textbf{\textit{{#1}}}}}
\newcommand{\CommentVarTok}[1]{\textcolor[rgb]{0.38,0.63,0.69}{\textbf{\textit{{#1}}}}}
\newcommand{\VariableTok}[1]{\textcolor[rgb]{0.10,0.09,0.49}{{#1}}}
\newcommand{\ControlFlowTok}[1]{\textcolor[rgb]{0.00,0.44,0.13}{\textbf{{#1}}}}
\newcommand{\OperatorTok}[1]{\textcolor[rgb]{0.40,0.40,0.40}{{#1}}}
\newcommand{\BuiltInTok}[1]{{#1}}
\newcommand{\ExtensionTok}[1]{{#1}}
\newcommand{\PreprocessorTok}[1]{\textcolor[rgb]{0.74,0.48,0.00}{{#1}}}
\newcommand{\AttributeTok}[1]{\textcolor[rgb]{0.49,0.56,0.16}{{#1}}}
\newcommand{\InformationTok}[1]{\textcolor[rgb]{0.38,0.63,0.69}{\textbf{\textit{{#1}}}}}
\newcommand{\WarningTok}[1]{\textcolor[rgb]{0.38,0.63,0.69}{\textbf{\textit{{#1}}}}}

% Define a nice break command that doesn't care if a line doesn't already
% exist.
\def\br{\hspace*{\fill} \\* }

% Math Jax compatability definitions
\def\gt{>}
\def\lt{<}

    \setcounter{secnumdepth}{5}

    % Colors for the hyperref package
    \definecolor{urlcolor}{rgb}{0,.145,.698}
    \definecolor{linkcolor}{rgb}{.71,0.21,0.01}
    \definecolor{citecolor}{rgb}{.12,.54,.11}

\DeclareTranslationFallback{Author}{Author}
\DeclareTranslation{Portuges}{Author}{Autor}

\DeclareTranslationFallback{List of Codes}{List of Codes}
\DeclareTranslation{Catalan}{List of Codes}{Llista de Codis}
\DeclareTranslation{Danish}{List of Codes}{Liste over Koder}
\DeclareTranslation{German}{List of Codes}{Liste der Codes}
\DeclareTranslation{Spanish}{List of Codes}{Lista de C\'{o}digos}
\DeclareTranslation{French}{List of Codes}{Liste des Codes}
\DeclareTranslation{Italian}{List of Codes}{Elenco dei Codici}
\DeclareTranslation{Dutch}{List of Codes}{Lijst van Codes}
\DeclareTranslation{Portuges}{List of Codes}{Lista de C\'{o}digos} 

\DeclareTranslationFallback{Supervisors}{Supervisors}
\DeclareTranslation{Catalan}{Supervisors}{Supervisors}
\DeclareTranslation{Danish}{Supervisors}{Vejledere}
\DeclareTranslation{German}{Supervisors}{Vorgesetzten}
\DeclareTranslation{Spanish}{Supervisors}{Supervisores}
\DeclareTranslation{French}{Supervisors}{Superviseurs}
\DeclareTranslation{Italian}{Supervisors}{Le autorit\`{a} di vigilanza}
\DeclareTranslation{Dutch}{Supervisors}{supervisors}
\DeclareTranslation{Portuguese}{Supervisors}{Supervisores} 

\definecolor{codegreen}{rgb}{0,0.6,0}
\definecolor{codegray}{rgb}{0.5,0.5,0.5}
\definecolor{codepurple}{rgb}{0.58,0,0.82}
\definecolor{backcolour}{rgb}{0.95,0.95,0.95}

\lstdefinestyle{mystyle}{
    commentstyle=\color{codegreen},
    keywordstyle=\color{magenta},
    numberstyle=\tiny\color{codegray},
    stringstyle=\color{codepurple},
    basicstyle=\ttfamily,
    breakatwhitespace=false,         
    keepspaces=true,                 
    numbers=left,                    
    numbersep=10pt,                  
    showspaces=false,                
    showstringspaces=false,
    showtabs=false,                  
    tabsize=2,
    breaklines=true,
    literate={\-}{}{0\discretionary{-}{}{-}},
  postbreak=\mbox{\textcolor{red}{$\hookrightarrow$}\space},
}

\lstset{style=mystyle} 

\surroundwithmdframed[
  hidealllines=true,
  backgroundcolor=backcolour,
  innerleftmargin=0pt,
  innerrightmargin=0pt,
  innertopmargin=0pt,
  innerbottommargin=0pt]{lstlisting}

 % Used to adjust the document margins
\usepackage{geometry}
\geometry{tmargin=1in,bmargin=1in,lmargin=1in,rmargin=1in,
nohead,includefoot,footskip=25pt}
% you can use showframe option to check the margins visually 

	% ensure new section starts on new page
	\addtokomafont{section}{\clearpage}

    % Prevent overflowing lines due to hard-to-break entities
    \sloppy 

    % Setup hyperref package
    \hypersetup{
      breaklinks=true,  % so long urls are correctly broken across lines
      colorlinks=true,
      urlcolor=urlcolor,
      linkcolor=linkcolor,
      citecolor=citecolor,
      }

    % ensure figures are placed within subsections
    \makeatletter
    \AtBeginDocument{%
      \expandafter\renewcommand\expandafter\subsection\expandafter
        {\expandafter\@fb@secFB\subsection}%
      \newcommand\@fb@secFB{\FloatBarrier
        \gdef\@fb@afterHHook{\@fb@topbarrier \gdef\@fb@afterHHook{}}}%
      \g@addto@macro\@afterheading{\@fb@afterHHook}%
      \gdef\@fb@afterHHook{}%
    }
    \makeatother

	% number figures, tables and equations by section
	\usepackage{chngcntr}
	\counterwithout{figure}{section}
	\counterwithout{table}{section}
	\counterwithout{equation}{section}
	\makeatletter
	\@addtoreset{table}{section}
	\@addtoreset{figure}{section}
	\@addtoreset{equation}{section}
	\makeatother
	\renewcommand\thetable{\thesection.\arabic{table}}
	\renewcommand\thefigure{\thesection.\arabic{figure}}
	\renewcommand\theequation{\thesection.\arabic{equation}}

        % set global options for float placement
        \makeatletter
          \providecommand*\setfloatlocations[2]{\@namedef{fps@#1}{#2}}
        \makeatother

    % align captions to left (indented)
	\captionsetup{justification=raggedright,
	singlelinecheck=false,format=hang,labelfont={it,bf}} 

	% shift footer down so space between separation line
	\ModifyLayer[addvoffset=.6ex]{scrheadings.foot.odd}
	\ModifyLayer[addvoffset=.6ex]{scrheadings.foot.even}
	\ModifyLayer[addvoffset=.6ex]{scrheadings.foot.oneside}
	\ModifyLayer[addvoffset=.6ex]{plain.scrheadings.foot.odd}
	\ModifyLayer[addvoffset=.6ex]{plain.scrheadings.foot.even}
	\ModifyLayer[addvoffset=.6ex]{plain.scrheadings.foot.oneside}
	\pagestyle{scrheadings}
	\clearscrheadfoot{}
	\ifoot{\leftmark}
	\renewcommand{\sectionmark}[1]{\markleft{\thesection\ #1}}
	\ofoot{\pagemark}
	\cfoot{}

% clereref must be loaded after anything that changes the referencing system
\usepackage{cleveref}
\creflabelformat{equation}{#2#1#3}

% make the code float work with cleverref
\crefname{codecell}{code}{codes}
\Crefname{codecell}{code}{codes}
% make the text float work with cleverref
\crefname{textcell}{text}{texts}
\Crefname{textcell}{text}{texts}
% make the text float work with cleverref
\crefname{errorcell}{error}{errors}
\Crefname{errorcell}{error}{errors}

	\begin{document}

		\begin{titlepage}

	\begin{center}

	\vspace*{1cm}

	\Huge\textbf{Visualisation of 3D atomic and electronic data in the Jupyter Notebook}

	\vspace{0.5cm}

	\vspace{1.5cm}

	\begin{minipage}{0.8\textwidth}   
		\begin{center}  
		\begin{minipage}{0.39\textwidth}
		\begin{flushleft} \Large
		\emph{\GetTranslation{Author}:}\\Chris Sewell\\
		\end{flushleft}
		\end{minipage}
		\hspace{\fill}
		\begin{minipage}{0.39\textwidth}
		\begin{flushright} \Large
		\end{flushright}
		\end{minipage}
		\end{center}   
	\end{minipage}

	\vfill

	\begin{minipage}{0.8\textwidth}
	\begin{center}
	\end{center} 
	\end{minipage}

	\vspace{0.8cm}
		  \LARGE{Imperial College London}\\
		  \LARGE{South Kensington, London}\\

	\vspace{0.4cm}

	\today

	\end{center}
	\end{titlepage}

		\begingroup
    \let\cleardoublepage\relax
    \let\clearpage\relax\tableofcontents\listoffigures\listoftables
    \endgroup

\section{Introduction}\label{introduction}

With the improvements in Jupyter Notebook, allowing for the synergy of
browser-side javascript and client-side python coding, it is becoming
possible to replicate the functionality of standalone atomic
visualisation packages (such as
\href{https://ovito.org/index.php/about}{ovito}). The added benefits
this approach brings is:

\begin{itemize}
\tightlist
\item
  Greater control and flexibility in the analysis and visualisation
  process
\item
  Fully autonomous replication of the analysis and visualisation
\item
  Better documentation of the analysis and visualisation
\end{itemize}

The goal of this notebook is to show a method for:

\begin{enumerate}
\def\labelenumi{\arabic{enumi}.}
\tightlist
\item
  reading/creating atomic configurations
\item
  visualising these in a Jupyter Notebook
\item
  Adding dynamic controls
\item
  Overlaying electronic level data (probability/spin densities)
\item
  Distributing the output
\end{enumerate}

The notebook and code used for this demonstration can be found
\href{https://github.com/chrisjsewell/chrisjsewell.github.io/tree/master/3d_atomic}{here}.

\section{A Quick Introduction to
JSON}\label{a-quick-introduction-to-json}

A recommended practice for structuring hierarchical data (which will be
used throughout this demonstration) is to use the JSON format because it
is:

\begin{itemize}
\tightlist
\item
  applicable for any (non-relational) data structure
\item
  lightweight and easy to read and edit
\item
  has a simple read/write mapping to python objects
\item
  widely used (especially in web technologies)
\end{itemize}

A good example of how this structure can being applied to quantum
chamical data can be found
\href{https://github.com/MolSSI/QC_JSON_Schema}{here} and, in fact, this
Jupyter Notebook itself is stored in JSON format:

\begin{lstlisting}[language={},postbreak={},numbers=none,xrightmargin=7pt,belowskip=5pt,aboveskip=5pt,breakindent=0pt]
cells:          [...]
metadata:       {...}
nbformat:       4
nbformat_minor: 2

\end{lstlisting}

Note that Python has a built-in \texttt{json} package, but we shall use
the \href{https://github.com/chrisjsewell/jsonextended}{jsonextended}
package, which has extended functionality for data parsing, manipulation
and visualisation.

\subsection{JSON Schema}\label{json-schema}

Rather than strictly controlling the entire data structure from the
outset, before acting on the data, we can simply validate that it
contains the required data keys and types required. This is in keeping
with the interpreted (as opposed to declarative) nature of Python.

Another advantage of json is that we can utilise the standard
\href{http://json-schema.org/}{json schema} approach to achieve this
validation. For example, a basic Jupyter Notebook schema would look like
this:

\begin{lstlisting}[language={},postbreak={},numbers=none,xrightmargin=7pt,belowskip=5pt,aboveskip=5pt,breakindent=0pt,escapechar=\%]
%\textcolor{ansi-blue}{properties}%: 
  %\textcolor{ansi-blue}{cells}%: 
    %\textcolor{ansi-blue}{items}%: 
      %\textcolor{ansi-blue}{properties}%: 
        %\textcolor{ansi-blue}{cell\_type}%: 
          %\textcolor{ansi-blue}{enum}%: [code, markdown]
          %\textcolor{ansi-blue}{type}%: string
      %\textcolor{ansi-blue}{required}%: [cell_type]
      %\textcolor{ansi-blue}{type}%: object
    %\textcolor{ansi-blue}{type}%: array
%\textcolor{ansi-blue}{required}%: [cells, metadata]
%\textcolor{ansi-blue}{type}%: object

\end{lstlisting}

JSON Schema basics are outlined
\href{https://spacetelescope.github.io/understanding-json-schema/basics.html}{here}
and a cheat sheet of standard keys can be found
\href{http://forivall.com/json-schema-cheatsheet/}{here}.

\section{Creating Atomic
Configurations}\label{creating-atomic-configurations}

The \href{http://pymatgen.org/}{pymatgen} package offers a means to
create/manipulate atomic configurations with repeating boundary
conditions. For this demonstration, we would like to select structures
from a folder of cif (crystallographic information files). This is
achieved by writing a \emph{plugin} for
\href{https://github.com/chrisjsewell/jsonextended}{jsonextended}.

\begin{lstlisting}[language={},postbreak={},numbers=none,xrightmargin=7pt,belowskip=5pt,aboveskip=5pt,breakindent=0pt]
FeS_greigite.cif: {...}
FeS_mackinawite.cif: {...}
FeS_marcasite.cif: {...}
FeS_pyrite.cif: {...}
FeS_pyrrhotite_4C_c2c.cif: {...}
FeS_troilite.cif: {...}
Fe_bcc.cif: {...}
S_8alpha_fddd.cif: {...}
S_8beta_p21c.cif: {...}

\end{lstlisting}

For each cif, we can now access a pymatgen structure:

\begin{lstlisting}[language={},postbreak={},numbers=none,xrightmargin=7pt,breakindent=0pt,aboveskip=5pt,belowskip=5pt]
Structure Summary
Lattice
    abc : 2.4820288072462011 2.4820288072462011 2.4820288072462011
 angles : 109.47122063449069 109.47122063449069 109.47122063449069
 volume : 11.770598948
      A : -1.4329999999999996 -1.4330000000000001 1.4329999999999998
      B : -1.4330000000000001 1.4330000000000001 -1.4330000000000001
      C : 1.4330000000000001 -1.4330000000000001 -1.4330000000000001
PeriodicSite: Fe (0.0000, 0.0000, 0.0000) [0.0000, 0.0000, 0.0000]
\end{lstlisting}

Note that the underlying structure of the pymatgen structure is JSON.

\begin{lstlisting}[language={},postbreak={},numbers=none,xrightmargin=7pt,belowskip=5pt,aboveskip=5pt,breakindent=0pt,escapechar=\%]
%\textcolor{ansi-blue}{@class}%: Structure
%\textcolor{ansi-blue}{@module}%: pymatgen.core.structure
%\textcolor{ansi-blue}{lattice}%: 
  %\textcolor{ansi-blue}{a}%: 2.48
  %\textcolor{ansi-blue}{alpha}%: 109.0
  %\textcolor{ansi-blue}{b}%: 2.48
  %\textcolor{ansi-blue}{beta}%: 109.0
  %\textcolor{ansi-blue}{c}%: 2.48
  %\textcolor{ansi-blue}{gamma}%: 109.0
  %\textcolor{ansi-blue}{matrix}%: [[-1.43, -1.43, 1.43], [-1.43, 
          1.43, -1.43], [1.43, -1.43, 
          -1.43]]
  %\textcolor{ansi-blue}{volume}%: 11.8
%\textcolor{ansi-blue}{sites}%: 
  %\textcolor{ansi-blue}{- abc}%: [0.0, 0.0, 0.0]
  %\textcolor{ansi-blue}{  label}%: Fe
  %\textcolor{ansi-blue}{  species}%: 
      %\textcolor{ansi-blue}{- element}%: Fe
      %\textcolor{ansi-blue}{  occu}%: 1
  %\textcolor{ansi-blue}{  xyz}%: [0.0, 0.0, 0.0]

\end{lstlisting}

\section{Preparation for
visualisation}\label{preparation-for-visualisation}

A visualisation requires the configuration to contain some additional
information, including the atom shape (e.g. sphere radius) and texture
(e.g. sphere color). Therefore, it will be helpful to create a view
agnostic data structure (i.e. independendant of any specific graphics
package) of all elements we wish to visualise. The key aspects of our
structure is:

\begin{enumerate}
\def\labelenumi{\arabic{enumi}.}
\tightlist
\item
  \textbf{elements}: A list of elements in the scene. For now we only
  have one type but more will be added later.
\item
  \textbf{transforms}: A list of geometric transforms we will apply
  globally (to all elements) and locally (to individual elements) before
  visualisation.
\end{enumerate}

We shall create a number of elements and transforms (and associated
validation schema) for this demonstration, but any number can be
specified.

\begin{lstlisting}[language={},postbreak={},numbers=none,xrightmargin=7pt,belowskip=5pt,aboveskip=5pt,breakindent=0pt,escapechar=\%]
%\textcolor{ansi-blue}{elements}%: 
  %\textcolor{ansi-blue}{- cell\_vectors}%: 
      %\textcolor{ansi-blue}{a}%: [3.67, 0.0, 2.25e-16]
      %\textcolor{ansi-blue}{b}%: [-2.25e-16, 3.67, 2.25e-16]
      %\textcolor{ansi-blue}{c}%: [0.0, 0.0, 5.03]
  %\textcolor{ansi-blue}{  centre}%: [1.84, 1.84, 2.52]
  %\textcolor{ansi-blue}{  color}%: #e06633
  %\textcolor{ansi-blue}{  coords}%: [[0.0, 0.0, 0.0], [1.84, 1.84, 2.25e-16]]
  %\textcolor{ansi-blue}{  label}%: Fe
  %\textcolor{ansi-blue}{  radius}%: 1.32
  %\textcolor{ansi-blue}{  sname}%: mackinawite
  %\textcolor{ansi-blue}{  transforms}%: []
  %\textcolor{ansi-blue}{  transparency}%: 1.0
  %\textcolor{ansi-blue}{  type}%: repeat_cell
  %\textcolor{ansi-blue}{- cell\_vectors}%: 
      %\textcolor{ansi-blue}{a}%: [3.67, 0.0, 2.25e-16]
      %\textcolor{ansi-blue}{b}%: [-2.25e-16, 3.67, 2.25e-16]
      %\textcolor{ansi-blue}{c}%: [0.0, 0.0, 5.03]
  %\textcolor{ansi-blue}{  centre}%: [1.84, 1.84, 2.52]
  %\textcolor{ansi-blue}{  color}%: #b2b200
  %\textcolor{ansi-blue}{  coords}%: [[-1.12e-16, 1.84, 1.31], [1.84, 0.0, 3.72]]
  %\textcolor{ansi-blue}{  label}%: S
  %\textcolor{ansi-blue}{  radius}%: 1.05
  %\textcolor{ansi-blue}{  sname}%: mackinawite
  %\textcolor{ansi-blue}{  transforms}%: []
  %\textcolor{ansi-blue}{  transparency}%: 1.0
  %\textcolor{ansi-blue}{  type}%: repeat_cell
%\textcolor{ansi-blue}{transforms}%: []

\end{lstlisting}

To create the repeat cells, we have deconstructed the pymatgen structure
and applied a mapping of atomic number to radius/color, using a
pre-constructed csv table (\cref{tbl:atom_map}).

    \begin{table}[H]\caption{The first rows of the atomic data lookup.}\label{tbl:atom_map}

\centering
\begin{adjustbox}{max width=\textwidth}
\begin{tabular}{lrrrrrrlrrrrl}
\toprule
{} &  Blue &  ElAffinity &  ElNeg &  Green &  Ionization &  Mass &      Name &   RBO &  RCov &  RVdW &   Red & Symbol \\
\midrule
  &       &             &        &        &             &       &           &       &       &       &       &        \\
1 &  0.75 &        0.75 &   2.20 &   0.75 &       13.60 &  1.01 &  Hydrogen &  0.31 &  0.31 &  1.10 &  0.75 &      H \\
2 &  1.00 &        0.00 &   0.00 &   1.00 &       24.59 &  4.00 &    Helium &  0.28 &  0.28 &  1.40 &  0.85 &     He \\
3 &  1.00 &        0.62 &   0.98 &   0.50 &        5.39 &  6.94 &   Lithium &  1.28 &  1.28 &  1.81 &  0.80 &     Li \\
\bottomrule
\end{tabular}

\end{adjustbox}
\end{table}

We have grouped the atoms with the similar visual representations,
rather than specifying each atom separately for simplicity. Here we do
this by atomic number, but equally it could be done by symmetry
equivalence or another metric.

\subsection{Geometric Transforms}\label{geometric-transforms}

Geometric transforms which we likely want to perform include:

\begin{itemize}
\tightlist
\item
  creating a supercell of the configuration
\item
  orientating the configuration in a convenient manner in the cartesian
  coordinate space
\item
  slicing into the configuration
\end{itemize}

\begin{lstlisting}[language={},postbreak={},numbers=none,xrightmargin=7pt,belowskip=5pt,aboveskip=5pt,breakindent=0pt,escapechar=\%]
%\textcolor{ansi-blue}{elements}%: 
  %\textcolor{ansi-blue}{- transforms}%: []
  %\textcolor{ansi-blue}{- transforms}%: []
%\textcolor{ansi-blue}{transforms}%: 
  %\textcolor{ansi-blue}{- cvector}%: b
  %\textcolor{ansi-blue}{  direction}%: (1, 0, 0)
  %\textcolor{ansi-blue}{  type}%: local_align
  %\textcolor{ansi-blue}{- cvector}%: a
  %\textcolor{ansi-blue}{  recentre}%: True
  %\textcolor{ansi-blue}{  rep}%: 2
  %\textcolor{ansi-blue}{  type}%: local_repeat
  %\textcolor{ansi-blue}{- centre}%: (0.0, 0.0, 0.0)
  %\textcolor{ansi-blue}{  type}%: recentre
  %\textcolor{ansi-blue}{- centre}%: None
  %\textcolor{ansi-blue}{  lbound}%: None
  %\textcolor{ansi-blue}{  normal}%: (0, 0, 1)
  %\textcolor{ansi-blue}{  type}%: slice
  %\textcolor{ansi-blue}{  ubound}%: 1.0

\end{lstlisting}

Note we haven't actually performed any of these transforms yet. This is
done done by creating a transformed copy of the data at visualisation
time.

\begin{lstlisting}[language={},postbreak={},numbers=none,xrightmargin=7pt,belowskip=5pt,aboveskip=5pt,breakindent=0pt,escapechar=\%]
%\textcolor{ansi-blue}{elements}%: 
  %\textcolor{ansi-blue}{- cell\_vectors}%: 
      %\textcolor{ansi-blue}{a}%: [14.7, 0.0, 9e-16]
      %\textcolor{ansi-blue}{b}%: [-9e-16, 14.7, 9e-16]
      %\textcolor{ansi-blue}{c}%: [0.0, 0.0, 20.1]
  %\textcolor{ansi-blue}{  centre}%: [0.0, 0.0, 0.0]
  %\textcolor{ansi-blue}{  color}%: #e06633
  %\textcolor{ansi-blue}{  coords}%: [[-7.35, -7.35, -10.1], [-5.51, -5.51, -10.1], [-3.67, -7.35, -10.1], ...(x61)]
  %\textcolor{ansi-blue}{  label}%: Fe
  %\textcolor{ansi-blue}{  radius}%: 1.32
  %\textcolor{ansi-blue}{  sname}%: mackinawite
  %\textcolor{ansi-blue}{  transforms}%: []
  %\textcolor{ansi-blue}{  transparency}%: 1.0
  %\textcolor{ansi-blue}{  type}%: repeat_cell
  %\textcolor{ansi-blue}{- cell\_vectors}%: 
      %\textcolor{ansi-blue}{a}%: [14.7, 0.0, 9e-16]
      %\textcolor{ansi-blue}{b}%: [-9e-16, 14.7, 9e-16]
      %\textcolor{ansi-blue}{c}%: [0.0, 0.0, 20.1]
  %\textcolor{ansi-blue}{  centre}%: [0.0, 0.0, 0.0]
  %\textcolor{ansi-blue}{  color}%: #b2b200
  %\textcolor{ansi-blue}{  coords}%: [[-7.35, -5.51, -8.76], [-5.51, -7.35, -6.34], [-3.67, -5.51, -8.76], ...(x29)]
  %\textcolor{ansi-blue}{  label}%: S
  %\textcolor{ansi-blue}{  radius}%: 1.05
  %\textcolor{ansi-blue}{  sname}%: mackinawite
  %\textcolor{ansi-blue}{  transforms}%: []
  %\textcolor{ansi-blue}{  transparency}%: 1.0
  %\textcolor{ansi-blue}{  type}%: repeat_cell
%\textcolor{ansi-blue}{transforms}%: []

\end{lstlisting}

\section{Visualising in the Jupyter
Notebook}\label{visualising-in-the-jupyter-notebook}

To create 3D renderings of the configuration, we will use
\href{http://ipyvolume.readthedocs.io}{ipyvolume} and its implementation
of the model/view pattern.

\begin{lstlisting}[language={},postbreak={},numbers=none,xrightmargin=7pt,breakindent=0pt,aboveskip=5pt,belowskip=5pt]
A Jupyter Widget
\end{lstlisting}

The rendering can also be captured as a screenshot or saved as an
image/html. We shall discuss in
Section \ref{publishing-and-distributing-analysis} how this can be
utilised to distribute the analysis.

\begin{lstlisting}[language={},postbreak={},numbers=none,xrightmargin=7pt,breakindent=0pt,aboveskip=5pt,belowskip=5pt]
A Jupyter Widget
\end{lstlisting}

\begin{figure}[H]\begin{center}\adjustimage{max size={0.9\linewidth}{0.9\paperheight},height=0.3\paperheight}{3D Atomic Visualisation_files/output_37_0.png}\end{center}\caption{an example of an ipyvolume scatter plot}\label{fig:ipyvol1}
    \end{figure}

\section{Adding Dynamic Controls}\label{adding-dynamic-controls}

\href{http://ipyvolume.readthedocs.io}{ipyvolume} utilises the
\href{http://ipywidgets.readthedocs.io}{ipywidgets} framework and thus
it is relatively trivial to set up dynamic controls.

\begin{lstlisting}[language={},postbreak={},numbers=none,xrightmargin=7pt,breakindent=0pt,aboveskip=5pt,belowskip=5pt]
A Jupyter Widget
\end{lstlisting}

We can bundle these in with the original container to create a bespoke
GUI.

\begin{lstlisting}[language={},postbreak={},numbers=none,xrightmargin=7pt,breakindent=0pt,aboveskip=5pt,belowskip=5pt]
A Jupyter Widget
\end{lstlisting}

\begin{lstlisting}[language={},postbreak={},numbers=none,xrightmargin=7pt,breakindent=0pt,aboveskip=5pt,belowskip=5pt]
A Jupyter Widget
\end{lstlisting}

\begin{figure}[H]\begin{center}\adjustimage{max size={0.9\linewidth}{0.9\paperheight},height=0.3\paperheight}{3D Atomic Visualisation_files/output_44_0.png}\end{center}\caption{an example of an ipyvolume scatter plot (with bespoke controls)}\label{fig:ipyvol2}
    \end{figure}

\section{Overlaying electronic level
data}\label{overlaying-electronic-level-data}

\emph{Ab initio} quantum simulation packages can compute electronic/spin
densities (to accompany the nuclei positions) in the form of a
discretized 3D cube. These can be overlayed onto the nuclei, by
\href{https://en.wikipedia.org/wiki/Volume_rendering}{volume rendering}
or \href{https://en.wikipedia.org/wiki/Isosurface}{isosurface} methods.

\subsection{Data Parsing}\label{data-parsing}

Taking the \href{http://www.crystal.unito.it/index.php}{CRYSTAL} program
as an example, output from the electronic density is principally output
into two files; one that contains the lattice vectors and nuclei
coordinates and one that contains a data cube of the electronic density,
with axis relating to the cell vectors. We can write parser plugins for
both these files:

\begin{lstlisting}[language={},postbreak={},numbers=none,xrightmargin=7pt,belowskip=5pt,aboveskip=5pt,breakindent=0pt,escapechar=\%]
%\textcolor{ansi-blue}{ech3.out}%: 
  %\textcolor{ansi-blue}{structure}%: Full Formula (Si2)
             Reduced Formula: Si
             abc   :   3.832519   3.832519   3.832519
             angles:  60.000000  60.000000  60.000000
             Sites (2)
               #  SP           a         b         c
             ---  ----  --------  --------  --------
               0  Si    0.125092  0.125092  0.125092
               1  Si    0.874908  0.875277  0.874908
%\textcolor{ansi-blue}{ech3\_dat.prop3d}%: 
  %\textcolor{ansi-blue}{charge\_density}%: np.array((100, 100, 100), min=2.68E-03, max=5.36E+02)
  %\textcolor{ansi-blue}{da\_vec}%: [ 0. 0.051729 0.051729]
  %\textcolor{ansi-blue}{db\_vec}%: [ 0.051729 0. 0.051729]
  %\textcolor{ansi-blue}{dc\_vec}%: [ 0.051729 0.051729 0. ]
  %\textcolor{ansi-blue}{na}%:    100
  %\textcolor{ansi-blue}{nb}%:    100
  %\textcolor{ansi-blue}{nc}%:    100
  %\textcolor{ansi-blue}{o\_vec}%: [0.0, 0.0, 0.0]

\end{lstlisting}

\subsection{Visualisation Preparation}\label{visualisation-preparation}

We then, follow the same process as for atoms; converting to a common
structure and adding geometric transforms. Note that we apply the slices
locally, so that we can first resize the density array. We do this
because, after the \texttt{repeat} transforms, the array size is now;
(400,400,400), which would be costly to compute the slice for. Resizing
by 0.25 reduces the array back to (100,100,100).

\begin{lstlisting}[language={},postbreak={},numbers=none,xrightmargin=7pt,belowskip=5pt,aboveskip=5pt,breakindent=0pt,escapechar=\%]
%\textcolor{ansi-blue}{elements}%: 
  %\textcolor{ansi-blue}{- transforms}%: 
      %\textcolor{ansi-blue}{- type}%: slice
  %\textcolor{ansi-blue}{  type}%: repeat_cell
  %\textcolor{ansi-blue}{- transforms}%: 
      %\textcolor{ansi-blue}{- sfraction}%: 0.25
      %\textcolor{ansi-blue}{  type}%: resize
      %\textcolor{ansi-blue}{- type}%: slice
  %\textcolor{ansi-blue}{  type}%: repeat_density
%\textcolor{ansi-blue}{transforms}%: 
  %\textcolor{ansi-blue}{- rep}%: 3
  %\textcolor{ansi-blue}{  type}%: local_repeat
  %\textcolor{ansi-blue}{- rep}%: 3
  %\textcolor{ansi-blue}{  type}%: local_repeat
  %\textcolor{ansi-blue}{- rep}%: 3
  %\textcolor{ansi-blue}{  type}%: local_repeat
  %\textcolor{ansi-blue}{- type}%: recentre

\end{lstlisting}

\subsection{Visualisation}\label{visualisation}

Before parsing to ipyvolume, we use the cell vectors to transform the
data cube into cartesian coordinates, such that voxels (cube sections)
outside of the cell volume are set as np.nan values. In the example, we
have thus created a really nice representation of the covalent bonding
in bulk silicon crystals.

\begin{lstlisting}[language={},postbreak={},numbers=none,xrightmargin=7pt,breakindent=0pt,aboveskip=5pt,belowskip=5pt]
A Jupyter Widget
\end{lstlisting}

\begin{lstlisting}[language={},postbreak={},numbers=none,xrightmargin=7pt,breakindent=0pt,aboveskip=5pt,belowskip=5pt]
A Jupyter Widget
\end{lstlisting}

\begin{figure}[H]\begin{center}\adjustimage{max size={0.9\linewidth}{0.9\paperheight},height=0.3\paperheight}{3D Atomic Visualisation_files/output_60_0.png}\end{center}\caption{an example of an ipyvolume volume and scatter plot (with bespoke controls)}\label{fig:ipyvol3}
    \end{figure}

\section{2D Representations}\label{d-representations}

Because the data is stored in a representation agnostic manner, this
allows for the possibility of displaying the data in multiple ways. In
particular, for publication quality images we may want to create a
2D-representation of the scene. Below we plot the atoms with a depth
perception effect, created by lightening the color of the atoms w.r.t
their depth into the page.

\begin{figure}[H]\begin{center}\adjustimage{max size={0.9\linewidth}{0.9\paperheight},height=0.2\paperheight}{3D Atomic Visualisation_files/output_63_0.pdf}\end{center}\caption{an example of a 2D representation of the atomic data}\label{fig:mpl1}
    \end{figure}

\section{Publishing and Distributing
Analysis}\label{publishing-and-distributing-analysis}

As discussed above, individual visualisations can be saved individually
as images or HTML. But a more complete solution is to use
\href{https://github.com/chrisjsewell/ipypublish}{ipypublish} to convert
the entire Jupyter Notebook to a document and/or presentation.
ipypublish utilises notebook/cell/output level metadata attributes to
define a greater level of control as to how elements in the notebook are
converted.

This entire document is a single notebook which is available as a
\href{https://github.com/chrisjsewell/chrisjsewell.github.io/blob/master/3d_atomic/3D\%20Atomic\%20Visualisation.ipynb}{Notebook},
\href{https://chrisjsewell.github.io/3d_atomic/converted/3D\%20Atomic\%20Visualisation.view_pdf.html}{PDF},
\href{https://chrisjsewell.github.io/3d_atomic/converted/3D\%20Atomic\%20Visualisation.html}{HTML}
or
\href{https://chrisjsewell.github.io/3d_atomic/converted/3D\%20Atomic\%20Visualisation.slides.html}{Reveal.JS
slideshow} document (click the hyperlinks to view them). This was
achieved by only the following command line commands:

\begin{verbatim}
$ nbpublish -pdf -ptemp -f latex_ipypublish_nocode "3D Atomic Visualisation.ipynb"
$ nbpublish -f html_ipypublish_all "3D Atomic Visualisation.ipynb"
$ nbpresent "3D Atomic Visualisation.ipynb"
\end{verbatim}

\section{TODO}\label{todo}

\begin{itemize}
\tightlist
\item
  Orthographic camera. Not yet implemented in ipyvolume, see
  \href{https://github.com/maartenbreddels/ipyvolume/issues/31}{this
  issue} for current status.
\item
  better control of spheres

  \begin{itemize}
  \tightlist
  \item
    exact control of radii (radius rather than scaling size)
  \item
    more segments (either direct control of segments, or a
    "sphere\_hi\_res" type)
  \item
    transparency level
  \item
    should color allow (r,g,b) tuple/array? because at the moment that
    doesn't work
  \end{itemize}
\item
  creation of array of arbitrary lines (like scatter but with;
  x0,y0,z0,x1,y1,z1)

  \begin{itemize}
  \tightlist
  \item
    show lattice bounding boxes: parallelepiped wire frames
  \item
    show bonds (i.e. connections) between different scatters
  \end{itemize}
\item
  show nearest-neighbour coordination: polygons with vertices at
  nearest-neighbour positions (as shown in \cref{fig:nnpolygons})
\end{itemize}

\begin{figure}[H]\begin{center}\adjustimage{max size={0.9\linewidth}{0.9\paperheight},height=0.2\paperheight}{3D Atomic Visualisation_files/output_68_0.jpeg}\end{center}\caption{an example of nearest-neighbour polygons}\label{fig:nnpolygons}
    \end{figure}

\begin{itemize}
\tightlist
\item
  functional (browser side) controls, e.g. slider to translate/rotate
  point set. Not yet implemented in ipywidgets, see
  \href{https://github.com/jupyter-widgets/ipywidgets/issues/1109}{this
  issue} for current status.
\item
  volumes:

  \begin{itemize}
  \tightlist
  \item
    RuntimeWarning: invalid value encountered in true\_divide
    (serialize.py:43) presumably for (0,0,0) gradients
  \item
    rarely get artifact rendering
  \item
    isosurface rendering
  \item
    multiple volumes in single plot
  \item
    volumes with arbitrary centres
  \item
    rotating volumes
  \end{itemize}
\item
  fullscreen

  \begin{itemize}
  \tightlist
  \item
    fails to open if multiple views instantiated
  \item
    if volume is present, then the rendering becomes very low resolution
    and, sometimes, the volume disappears completely on exit
  \end{itemize}
\item
  2d volume representation

  \begin{itemize}
  \tightlist
  \item
    define slice into cube and use matplotlib.contour
  \end{itemize}
\end{itemize}

	\end{document}

